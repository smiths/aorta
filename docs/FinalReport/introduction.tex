\chapter{Introduction}
This chapter includes an introduction for AortaGeomRecon assurance cases project. 

%Every thesis needs an introductory chapter
%
%While you're here, you need to go into \texttt{definitions.tex} to set all the 
%information needed for the front matter (e.g. title, author) and page 
%header/footer.
%
%You will also find the School of Graduate Studies' preparation guide (August 
%2021) for theses and reports. I would give it a quick read so you know what's 
%expected.

\section{Objective}
The main goals of this project starts with building a software that can quickly build the 3D geometry of the Aorta from CT Chest scans, while applying the assurance cases for this academic medical software. This project shows the assurance cases can indeed help build up our confidence in the medial software in general, because medical software like real-time system software, needed completeness and correctness.


\section{Background}
Aorta  \\
The largest artery that carries blood from the heart to the circulatory system.\\
Aorta segmentation in CT scans is important for:
\begin{itemize}
\item Coarctation of the aorta
\item Aortic calcification quantification
\item To guide the segmentation of other central vessels. 
\end{itemize} ~\\
Assurance cases\\
3D Slicer\\
Image Processing\\

\section{Problem Statement}
Build Software and Assurance cases for this software.
Start with a list of Functional and Non-Functional requirements.
