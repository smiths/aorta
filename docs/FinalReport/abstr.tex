\prefacesection{Abstract}

Assurance cases have been used to build safe real-time system software. Like real-time systems, medical image processing software also requires a high standard of correctness, completeness, and consistency. Therefore, we have investigated adopting real-time assurance case techniques to medical image processing software. For this project, we develop medical image processing software, called AortaGeomRecon, that helps to diagnose issues related to the aorta. We build an assurance case for this software by expanding on the evidence used for previous work on assurance cases for scientific software. In this way, we reinforce our confidence in the AortaGeomRecon's correctness, completeness, and consistency. Our techniques can be generalized to other scientific software.

With a Jupyter Notebook program left by a previous student as a starting point, we improved the existing segmentation algorithm and developed a 3D Slicer extension module that includes a Graphical User Interface and a module for input parameters. We built the continuous integration infrastructure with GitHub Actions. This allows us to update the algorithm while ensuring that it is at least as good as its previous version. Additionally, a linter is set up as part of the continuous integration process to ensure the program's readability by enforcing the PEP8 standard.

Next, we built assurance cases in Goal Structuring Notation with the bottom-up approaches. We gathered our existing evidence and explored new implementation requirements for the new evidence. We finalized our documentation on requirements, system architecture, user manual, and detailed design, which reinforces our confidence in these documentations and that the design compies with the documentation. Finally, we conduct a code review meeting and an algorithm review meeting to discuss our methodologies. These sessions provided insights into potential enhancements for our segmentation algorithm and affirmed the correctness of our remaining methods.