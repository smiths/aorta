\chapter{Conclusion and Future Works}

In this chapter, we provide a summary of the thesis (section \ref{thesis_sum}), the challenges(section \ref{challenge}), and the future work (section \ref{fw}).

\section{Thesis Summary}\label{thesis_sum}

In this project, we developed a software as a 3D Slicer extension to semi-automatically extract the 3D geometry of the aorta. To build confidence in the software, we applied assurance case arguments. The project started from a Jupyter Notebook program as left by a previous student, with this as a starting point, we explored . 

%We have then developed an idea of developing a software that is first conveniant to use, at least on gathering parameters. This lead to developing a 3D Slicer extension module, because 3D Slicer already provides useful features such as Volume Rendering, Volume visualization, Crop Volume, etc. While building the software, we build the assurance cases in Goal Structuring Notation with the bottom-up approaches; we keep ask ourself what are the evidences that are necessary to support our claim in requirements, implementation, operational assumptions and inputs assumptions? Keep that question in mind, we have finalized our SRS document, Module Guide document, wrote user instructions, built design document in HTML and publish it on a website, and linking all assurance cases to support our arguments. Finally, we have used GitHub's features for Continuous Integration infrastructure, and project management. We have built 2 automated process that act as a linter and continuous integration tests, these processes helped a lot in detecting bugs and errors in implementation. As for project management, GitHub Issues, Discussions, and Pull requests were used throughout the development of the software. Me and Dr. Spencer Smith were able to keep up good communication through these features.

\section{Challenge}\label{challenge}
In the course of this project, we have summarized a list of challenges and tasks that we could have done better. The first challenge was looking for an ideal platform to develop \progname{} software. Until the point where we see that it is nearly impossible to build from scratch a volume visualization system like the volume visualization provided by 3D Slicer, time and efforts have been wasted in design a UI, finding the right tool to build the UI, etc. 3D Slicer itself is a very complex software, the development resource is limited and difficult to understand.

Another obstacle that we have is having a domain expert to examine the quality of our segmentation result and other documentation. This medical software's intended user is a university student studying in medical science or medicine, who likes to get an aorta's image or quantified volume. Throughout the development of the \progname{}, we did not have an intended user or a domain expert to review our software. However, me and Dr. Spencer Smith were also lacking the knowledge and do not know the expectation of the intender user or domain expert, this causes ambiguity in understanding the true requirements of \progname{}.

Finally, it is very challenging of understanding assurance cases with a limited time, and building the assurance cases for \progname{} was unclear for me. Gathering the evidences and support our arguments was not in my imagination at the beginning of the project, without truely understanding our goals of the project, I was not certain what I was really doing for this project. Until we have several pieces linking together, I was finally understanding and making more efforts in the right direction. 

\section{Future Works}\label{fw}

In this section, we will discuss some possible future works that can continue to make \progname{} better. The first improvement can be done in segmentation algorithm, and the second improvement can finalize our assurance case.

\subsection{Segmentation Algorithm}

In this paper \cite{6346433}, we were able to discover a new segmentation algorithm that also needs a cropped volume and the aorta seeds to perform segmentation. However, it required less hyperparameters such as the parameters for SimpleITK's ThresholdSegmentationLevelSetsFilter. Using this algorithm effectively reduces the number of hyperparameters, which lead to a better and safer segmentation results.

\subsection{Assurance Case}

There is room for improvements on the arguments of the requirements of \progname{}. The correctness of the document is reviewed and approved by a domain expert, where there should be evidences that can support the argument.