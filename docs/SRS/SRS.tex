\documentclass[12pt]{article}

\usepackage{amsmath, mathtools}
\usepackage{amsfonts}
\usepackage{amssymb}
\usepackage{graphicx}
\usepackage{colortbl}
\usepackage{xr}
\usepackage{hyperref}
\usepackage{longtable}
\usepackage{xfrac}
\usepackage{tabularx}
\usepackage{float}
\usepackage{siunitx}
\usepackage{booktabs}
\usepackage{caption}
\usepackage{pdflscape}
\usepackage{afterpage}
\usepackage{subfig}

\usepackage[round]{natbib}

%\usepackage{refcheck}

\hypersetup{
    bookmarks=true,         % show bookmarks bar?
      colorlinks=true,       % false: boxed links; true: colored links
    linkcolor=red,          % color of internal links (change box color with linkbordercolor)
    citecolor=green,        % color of links to bibliography
    filecolor=magenta,      % color of file links
    urlcolor=cyan           % color of external links
}

\input{../Comments}
%% Common Parts

\newcommand{\progname}{AortaGeomRecon} % PUT YOUR PROGRAM NAME HERE
\newcommand{\authname}{Jingyi Lin} % AUTHOR NAMES                  

\usepackage{hyperref}
    \hypersetup{colorlinks=true, linkcolor=blue, citecolor=blue, filecolor=blue,
                urlcolor=blue, unicode=false}
    \urlstyle{same}
                                


% For easy change of table widths
\newcommand{\colZwidth}{1.0\textwidth}
\newcommand{\colAwidth}{0.13\textwidth}
\newcommand{\colBwidth}{0.82\textwidth}
\newcommand{\colCwidth}{0.1\textwidth}
\newcommand{\colDwidth}{0.05\textwidth}
\newcommand{\colEwidth}{0.8\textwidth}
\newcommand{\colFwidth}{0.17\textwidth}
\newcommand{\colGwidth}{0.5\textwidth}
\newcommand{\colHwidth}{0.28\textwidth}

% Used so that cross-references have a meaningful prefix
\newcounter{defnum} %Definition Number
\newcommand{\dthedefnum}{GD\thedefnum}
\newcommand{\dref}[1]{GD\ref{#1}}
\newcounter{datadefnum} %Datadefinition Number
\newcommand{\ddthedatadefnum}{DD\thedatadefnum}
\newcommand{\ddref}[1]{DD\ref{#1}}
\newcounter{theorynum} %Theory Number
\newcommand{\tthetheorynum}{T\thetheorynum}
\newcommand{\tref}[1]{T\ref{#1}}
\newcounter{tablenum} %Table Number
\newcommand{\tbthetablenum}{T\thetablenum}
\newcommand{\tbref}[1]{TB\ref{#1}}
\newcounter{assumpnum} %Assumption Number
\newcommand{\atheassumpnum}{P\theassumpnum}
\newcommand{\aref}[1]{A\ref{#1}}
\newcounter{goalnum} %Goal Number
\newcommand{\gthegoalnum}{P\thegoalnum}
\newcommand{\gsref}[1]{GS\ref{#1}}
\newcounter{instnum} %Instance Number
\newcommand{\itheinstnum}{IM\theinstnum}
\newcommand{\iref}[1]{IM\ref{#1}}
\newcounter{reqnum} %Requirement Number
\newcommand{\rthereqnum}{P\thereqnum}
\newcommand{\rref}[1]{R\ref{#1}}
\newcounter{nfrnum} %NFR Number
\newcommand{\rthenfrnum}{NFR\thenfrnum}
\newcommand{\nfrref}[1]{NFR\ref{#1}}
\newcounter{lcnum} %Likely change number
\newcommand{\lthelcnum}{LC\thelcnum}
\newcommand{\lcref}[1]{LC\ref{#1}}

\usepackage{fullpage}

\newcommand{\deftheory}[9][Not Applicable]
{
\newpage
\noindent \rule{\textwidth}{0.5mm}

\paragraph{RefName: } \textbf{#2} \phantomsection 
\label{#2}

\paragraph{Label:} #3

\noindent \rule{\textwidth}{0.5mm}

\paragraph{Equation:}

#4

\paragraph{Description:}

#5

\paragraph{Notes:}

#6

\paragraph{Source:}

#7

\paragraph{Ref.\ By:}

#8

\paragraph{Preconditions for \hyperref[#2]{#2}:}
\label{#2_precond}

#9

\paragraph{Derivation for \hyperref[#2]{#2}:}
\label{#2_deriv}

#1

\noindent \rule{\textwidth}{0.5mm}

}

\begin{document}

\title{Software Requirements Specification for \progname{}} 
\author{\authname}
\date{\today}
	
\maketitle

~\newpage

\pagenumbering{roman}

\tableofcontents

~\newpage

\section*{Revision History}

\begin{tabularx}{\textwidth}{p{3cm}p{2cm}X}
\toprule {\bf Date} & {\bf Version} & {\bf Notes}\\
\midrule
2023-02-12 & 1.0 & Notes\\
\midrule
2023-03-01 & 1.01 & Modified system context image, coordinate systems, and goal statements.\\
\midrule
2023-04-29 & 1.02 & Added requirements, instance models, data definitions\\
\bottomrule
\end{tabularx}

~\newpage

\section{Reference Material}

This section records information for easy reference.

\subsection{Table of Units}

Throughout this document SI (Syst\`{e}me International d'Unit\'{e}s) is employed
as the unit system.  In addition to the basic units, several derived units are
used as described below.  For each unit, the symbol is given followed by a
description of the unit and the SI name.
~\newline

%\renewcommand{\arraystretch}{1.2}
%\begin{table}[ht]
  % \noindent \begin{tabular}{l l l} 
  %   \toprule		
  %   \textbf{symbol} & \textbf{unit} & \textbf{SI}\\
  %   \midrule 
  %   \si{\metre} & length & metre\\
  %   \si{\kilogram} & mass	& kilogram\\
  %   \si{\second} & time & second\\
  %   \si{\celsius} & temperature & centigrade\\
  %   \si{\joule} & energy & joule\\
  %   \si{\watt} & power & watt (W = \si{\joule\per\second})\\
  %   \bottomrule
  % \end{tabular}
  %	\caption{Provide a caption}
%\end{table}

\subsection{Table of Symbols}

The table that follows summarizes the symbols used in this document along with
their units.  The choice of symbols was made to be consistent with existing documentation for 3D Slicer program. 
The symbols are listed in alphabetical order.

\renewcommand{\arraystretch}{1.2}
\noindent \begin{longtable}{l p{2cm} p{12cm}} \toprule
  \textbf{symbol} & \textbf{type} & \textbf{description}\\
  \midrule
\\
\textit{LOW} &  $\mathbb{N}$ &   A low intensity values means 0 on a scale of 0 and 1, or 0 on a scale of 0 to 255.
\\ 
\textit{HIGH} &  $\mathbb{N}$ &  A high intensity values means 1 on a scale of 0 and 1, or 255 on a scale of 0 to 255.
\\ 
\textit{slice} &  $\mathbb{R}^{m \times n}$ &  A slice is a 2 dimensional image view from the superior to inferior direction.
\\ 
\textit{Sv} & $\mathbb{N}^{3}$ &  A coordinate indicates the indexes of a starting voxel.
\\
\textit{S} & $\mathbb{N}^{3}$ &  The size to crop a rectangular shaped subset of a volume.
\\
\textit{v} &  $\mathbb{R}$ &  A voxel reports the intensity of a single point on a grey-scale 3 dimensional image.
\\
\textit{volume} & $\mathbb{R}^{m \times n \times p}$ &  Volume formed by a sequence of slice
\\

\bottomrule
\caption{Table of Symbols}
\label{Table of Symbols}
\end{longtable}

% \renewcommand{\arraystretch}{1.2}
% %\noindent \begin{tabularx}{1.0\textwidth}{l l X}
% \noindent \begin{longtable*}{l l p{12cm}} \toprule
% \textbf{symbol} & \textbf{unit} & \textbf{description}\\
% \midrule 
% $A_C$ & \si[per-mode=symbol] {\square\metre} & coil surface area
% \\
% $A_\text{in}$ & \si[per-mode=symbol] {\square\metre} & surface area over 
% which heat is transferred in
% \\ 
% \bottomrule
% \end{longtable*}
% \plt{Use your problems actual symbols.  The si package is a good idea to use for
%   units.}

\subsection{Abbreviations and Acronyms}

\renewcommand{\arraystretch}{1.2}
\begin{tabular}{l l} 
  \toprule		
  \textbf{symbol} & \textbf{description}\\
  \midrule 
  A & Assumption\\
  \progname & Aorta Geometry Reconstructor\\
  DD & Data Definition\\
  DICOM & Digital Imaging and Communications in Medicine\\
  GD & General Definition\\
  GS & Goal Statement\\
  IM & Instance Model\\
  LC & Likely Change\\
  PS & Physical System Description\\
  R & Requirement\\
  SRS & Software Requirements Specification\\
  T & Theoretical Model\\
  \bottomrule
\end{tabular}\\

% \subsection{Mathematical Notation}

% \plt{This section is optional, but should be included for projects that make use
%   of notation to convey mathematical information.  For instance, if typographic
%   conventions (like bold face font) are used to distinguish matrices, this
%   should be stated here.  If symbols are used to show mathematical operations,
%   these should be summarized here.  In some cases the easiest way to summarize
%   the notation is to point to a text or other source that explains the
%   notation.}

% \plt{This section was added to the template because some students use very
%   domain specific notation.  This notation will not be readily understandable to
%   people outside of your domain.  It should be explained.}

\newpage

\pagenumbering{arabic}

% \plt{This SRS template is based on \citet{SmithAndLai2005, SmithEtAl2007}.  It
%   will get you started.  You should not modify the section headings, without
%   first discussing the change with the course instructor.  Modification means
%   you are not following the template, which loses some of the advantage of a
%   template, especially standardization.  Although the bits shown below do not
%   include type information, you may need to add this information for your
%   problem.  If you are unsure, please can ask the instructor.}

% \plt{Feel free to change the appearance of the report by modifying the LaTeX
%   commands.}

% \plt{This template document assumes that a single program is being documented.
%   If you are documenting a family of models, you should start with a commonality
%   analysis.  A separate template is provided for this.  For program
%   families you should look at \cite{Smith2006, SmithMcCutchanAndCarette2017}.
%   Single family member programs are often programs based on a single physical
%   model.  General purpose tools are usually documented as a family.  Families of
%   physical models also come up.}

% \plt{The SRS is not generally written, or read, sequentially.  The SRS is a
%   reference document.  It is generally read in an ad hoc order, as the need
%   arises.  For writing an SRS, and for reading one for the first time, the
%   suggested order of sections is:
% \begin{itemize}
% \item Goal Statement
% \item Instance Models
% \item Requirements
% \item Introduction
% \item Specific System Description
% \end{itemize}
% }

% \plt{Guiding principles for the SRS document:
% \begin{itemize}
% \item Do not repeat the same information at the same abstraction level.  If
%   information is repeated, the repetition should be at a different abstraction
%   level.  For instance, there will be overlap between the scope section and the
%   assumptions, but the scope section will not go into as much detail as the
%   assumptions section.
% \end{itemize}
% }

% \plt{The template description comments should be disabled before submitting this
%   document for grading.}

% \plt{You can borrow any wording from the text given in the template.  It is part
%   of the template, and not considered an instance of academic integrity.  Of
%   course, you need to cite the source of the template.}

% \plt{When the documentation is done, it should be possible to trace back to the
%   source of every piece of information.  Some information will come from
%   external sources, like terminology.  Other information will be derived, like
%   General Definitions.}

% \plt{An SRS document should have the following qualities: unambiguous,
%   consistent, complete, validatable, abstract and traceable.}

% \plt{The overall goal of the SRS is that someone that meets the Characteristics
%   of the Intended Reader (Section~\ref{sec_IntendedReader}) can learn,
%   understand and verify the captured domain knowledge.  They should not have to
%   trust the authors of the SRS on any statements.  They should be able to
%   independently verify/derive every statement made.}

\section{Introduction}

This document provides an overview of the Software Requirements Specification
(SRS) for the \progname{}. \progname{}
provides a highly customizable aorta segmentation module, 
and an interactive user interface to apply the segmentation workflow.

% \plt{The introduction section is written to introduce the problem.  It starts
%   general and focuses on the problem domain. The general advice is to start with
% a paragraph or two that describes the problem, followed by a ``roadmap''
% paragraph.  A roadmap orients the reader by telling them what sub-sections to
% expect in the Introduction section.}

\subsection{Purpose of Document}
The main purpose of this document is to provide sufficient information to
understand what \progname{} module does. The goals and theoretical models used in the
\progname{} segmentation module implementation are provided,
with an emphasis on explicitly identifying assumptions and unambiguous
definitions.

% \plt{This section summarizes the purpose of the SRS document.  It does not focus
%   on the problem itself.  The problem is described in the ``Problem
%   Description'' section (Section~\ref{Sec_pd}).  The purpose is for the document
%   in the context of the project itself, not in the context of this course.
%   Although the ``purpose'' of the document is to get a grade, you should not
%   mention this.  Instead, ``fake it'' as if this is a real project.  The purpose
%   section will be similar between projects.  The purpose of the document is the
%   purpose of the SRS, including communication, planning for the design stage,
%   etc.}

\subsection{Scope of Requirements} 

% \plt{Modelling the real world requires simplification.  The full complexity of
%   the actual physics, chemistry, biology is too much for existing models, and
%   for existing computational solution techniques.  Rather than say what is in
%   the scope, it is usually easier to say what is not.  You can think of it as
%   the scope is initially everything, and then it is constrained to create the
%   actual scope.  For instance, the problem can be restricted to 2 dimensions, or
%   it can ignore the effect of temperature (or pressure) on the material
%   properties, etc.}  

% \plt{The scope section is related to the assumptions section
%   (Section~\ref{sec_assumpt}).  However, the scope and the assumptions are not
%   at the same level of abstraction.  The scope is at a high level.  The focus is
%   on the ``big picture'' assumptions.  The assumptions section lists, and
%   describes, all of the assumptions.}

% \subsection{Characteristics of Intended Reader} \label{sec_IntendedReader}

% \plt{This section summarizes the skills and knowledge of the readers of the
%   SRS.  It does NOT have the same purpose as the ``User Characteristics''
%   section (Section~\ref{SecUserCharacteristics}).  The intended readers are the
%   people that will read, review and maintain the SRS.  They are the people that
%   will conceivably design the software that is intended to meet the
%   requirements.  The user, on the other hand, is the person that uses the
%   software that is built.  They may never read this SRS document.  Of course,
%   the same person could be a ``user'' and an ``intended reader.''}

% \plt{The intended reader characteristics should be written as unambiguously and
%   as specifically as possible.  Rather than say, the user should have an
%   understanding of physics, say what kind of physics and at what level.  For
%   instance, is high school physics adequate, or should the reader have had a
%   graduate course on advanced quantum mechanics?}

\subsection{Organization of Document}

%\plt{This section provides a roadmap of the SRS document.  It will help the
%  reader orient themselves.  It will provide direction that will help them
%  select which sections they want to read, and in what order.  This section will
%  be similar between project.}

The organization of this document follows the template for an SRS for
scientific computing software proposed by \cite{Koothoor2013} and
\cite{SmithAndLai2005}. The presentation
follows the standard pattern of presenting goals, theories, definitions and
assumptions. The goal statements are refined to the theoretical models, and
theoretical
models to the instance models. For readers that would like a more bottom-up
approach, they can start reading the instance models in Section
\ref{sssec:im} and trace back to find any additional information they
require. 

\section{General System Description}

This section provides general information about the system.  It identifies the
interfaces between the system and its environment, describes the user
characteristics and lists the system constraints.  

\subsection{System Context}
Figure \ref{Fig_SystemContext} shows the system context.  A circle represents an
external entity outside the software, the user in this case.  A rectangle
represents the software system itself.  Arrows are used to show the data
flow between the system and its environment. \\ \\ \\ \\ \\ \\ \\


\begin{figure}[h!]
\begin{center}
 \includegraphics[width=1\textwidth]{AortaSystemContext.png}
\caption{System Context}
\label{Fig_SystemContext} 
\end{center}
\end{figure}


\begin{itemize}
\item User Responsibilities:
\begin{itemize}
\item Provide the input data to the system
\item Ensure the input meets the necessary assumptions
\item Verify the result meets their requirements, otherwise repeat the process with a different seed values.
\end{itemize}
\item \progname{} Responsibilities:
\begin{itemize}
\item Provide DICOM data reader which can take a path to a folder containing DICOM files.
\item Provide crop functionality to easily select a region of interest. 
\item Provide simple interactions to obtain and store the users' inputs. This includes a data probe to read voxel location which stored as a coordinate, and text inputs for real numbers.
\item Provide visualization on the result data.
\end{itemize}
\end{itemize}

\subsection{User Characteristics} \label{SecUserCharacteristics}
The end user of \progname{} have taking the university level anatomy introduction course, and is capable to point out the center of the descending aorta and the ascending aorta.

% \plt{This section summarizes the knowledge/skills expected of the user.
%   Measuring usability, which is often a required non-function requirement,
%   requires knowledge of a typical user.  As mentioned above, the user is a
%   different role from the ``intended reader,'' as given in
%   Section~\ref{sec_IntendedReader}.  As in Section~\ref{sec_IntendedReader}, the
%   user characteristics should be specific an unambiguous.  For instance, ``The
%   end user of \progname{} should have an understanding of undergraduate Level 1
%   Calculus and Physics.''}

\subsection{System Constraints}
% Intended environment to run the program on are the Windows 10 systems.
% \plt{System constraints differ from other type of requirements because they
%   limit the developers' options in the system design and they identify how the
%   eventual system must fit into the world. This is the only place in the SRS
%   where design decisions can be specified.  That is, the quality requirement for
%   abstraction is relaxed here.  However, system constraints should only be
%   included if they are truly required.}

\section{Specific System Description}

This section first presents the problem description, which gives a high-level
view of the problem to be solved.  This is followed by the solution characteristics
specification, which presents the assumptions, theories, definitions and finally
the instance models.  

% \plt{Add any project specific details that are relevant
%   for the section overview.}

\subsection{Problem Description} \label{Sec_pd}

The main purpose of \progname{} is to semi-automatically segment a 3D aorta geometry from a chest CT scan.

\subsubsection{Organ segmentation}
The organ segmentation or the organ boundary is useful for orientation and identification of the regions of interests inside the organ during the diagnostic or treatment procedure. The aorta segmentation is important for aortic calcification quantification and to guide the segmentation of other central vessels.

\subsubsection{Coordinate Systems}
This subsection provides a list of terms that are used in the subsequent
sections and their meaning, with the purpose of reducing ambiguity and making it
easier to correctly understand the requirements. \\

\noindent While working with medical images, it is necessary to be familiar with the different coordinate systems of the medical literarure and how data (voxels' orientation) is interpreted in different medical and non-medical software. Each coordinate system uses one or more numbers (coordinates) to uniquely determine the position of a point (in the medical context, we refer to each point as a voxel). The purpose of this section is to introduce some of the coordinate systems related to the medical imaging. There are different coordinate systems to represent data. A knowledge of the following coordinate systems is needed to work with the medical images.

\paragraph{Cartesian Coordinate System}
A Cartesian coordinate system is a coordinate system that specifies each point uniquely in a 2D plane by a pair of numerical coordinates or in a 3D space by three numerical coordinates. We assume a right-hand Cartesian coordinate system throughout this document.

\paragraph{World Coordinate System}
World Coordinate System (WCS) is a Cartesian coordinate system that describes the physical coordinates associated with a model such as an MRI scanner or a patient. While each model has its own coordinate system, without a universal coordinate system such as WCS, they cannot interact with each other. For model interaction to be possible, their coordinate systems must be transformed into the WCS. Figure \ref{WCS} shows the WCS corresponding space and axes.
\begin{figure}[hbpt!]
\centering
\includegraphics[width=50mm]{worldCoordinateSystem.png}
\caption{World Coordinate System Space and Axes~\cite{slicerWCS}}
\label{WCS}
\end{figure}
\paragraph{Anatomical Coordinate System}
Anatomical coordinate system, also known as patient coordinate system, is a right-handed 3D coordinate system that describes the standard anatomical position of a human using the following 3 orthogonal planes:
\begin{itemize}
\item{Axial / Transverse plane:} is a plane parallel to the ground that separates the body into head (superior) and tail (inferior) positions.
\item{Coronal / Frontal plane:} is a plane perpendicular to the ground that divides the body into front (anterior) and back (posterior) positions.
\item{Sagittal / Median plane:} is a plane that divides the body into right and left positions.
\end{itemize}
Figure \ref{ACS} shows this coordinate system.

\begin{figure}[htbp!] 
\centering
\subfloat{
\includegraphics[width=80mm]{anatomicalCoordinateSystem1.png}
}

\subfloat{
\includegraphics[width=60mm]{anatomicalCoordinateSystem2.png}
}
\caption{Anatomical Coordinate System Space and Axes~\cite{slicerWCS}} %without this we cannot have label
\label{ACS}
\end{figure}
\indent

Medical applications follow an anatomical coordinate system to store voxels in sequences. Depending on how the data is stored, this coordinate system can be divided into different bases. The most common ones are:
\begin{itemize}
\item{LPS Coordinate System:}

The LPS coordinate system is used in DICOM images and by the ITK toolkit. In this system, voxels are ordered from left to right in a row, rows are ordered from posterior to anterior, and slices are stored from inferior to superior.
\indent
LPS stands for Left-Posterior-Superior which indicates the directions that spatial axes are increasing.
% RAI Radiological preferred axes.
\item{RAS Coordinate System:}

The RAS coordiante system is the preferred basis for Neurological applications such as 3dfim+, and 3D Slicer. RAS stands for Right-Anterior-Superior is similar to LPS with the first two axes flipped.
\end{itemize}

\paragraph{Image Coordinate System}
To specify locations in an image we need to know to which coordinate system it is referenced. Different software may use different orders as their index convention.


\subsubsection{Physical System Description} \label{sec_phySystDescrip}

% \plt{The purpose of this section is to clearly and unambiguously state the
%   physical system that is to be modelled. Effective problem solving requires a
%   logical and organized approach. The statements on the physical system to be
%   studied should cover enough information to solve the problem. The physical
%   description involves element identification, where elements are defined as
%   independent and separable items of the physical system. Some example elements
%   include acceleration due to gravity, the mass of an object, and the size and
%   shape of an object. Each element should be identified and labelled, with their
%   interesting properties specified clearly. The physical description can also
%   include interactions of the elements, such as the following: i) the
%   interactions between the elements and their physical environment; ii) the
%   interactions between elements; and, iii) the initial or boundary conditions.}

We do not study the physical system for DICOM or how the data is actually generated.

% \begin{itemize}

% \item[PS1:] 

% \item[PS2:] ...

% \end{itemize}

% \plt{A figure here makes sense for most SRS documents}

% \begin{figure}[h!]
% \begin{center}
% %\rotatebox{-90}
% {
%  \includegraphics[width=0.5\textwidth]{<FigureName>}
% }
% \caption{\label{<Label>} <Caption>}
% \end{center}
% \end{figure}

\subsubsection{Goal Statements}

% \plt{The goal statements refine the ``Problem Description''
%   (Section~\ref{Sec_pd}).  A goal is a functional objective the system under
%   consideration should achieve. Goals provide criteria for sufficient
%   completeness of a requirements specification and for requirements
%   pertinence. Goals will be refined in Section “Instanced Models”
%   (Section~\ref{sec_instance}). Large and complex goals should be decomposed
%   into smaller sub-goals.  The goals are written abstractly, with a minimal
%   amount of technical language.  They should be understandable by non-domain
%   experts.}

\noindent Given the DICOM image that includes patient's chest,
the descending aorta center voxel coordinate,
and the ascending aorta center voxel coordinate, the goal statements is:

\begin{itemize}

\item[GS\refstepcounter{goalnum}\thegoalnum \label{Gsegment}:] 
Extract the three dimensional segmentation of the aorta.

\end{itemize}


\subsubsection{Assumptions} \label{sec_assumpt}

This section simplifies the original problem and helps in developing the
theoretical model by filling in the missing information for the physical
system. The numbers given in the square brackets refer to the theoretical model
[T], general definition [GD], data definition [DD], instance model [IM], or
likely change [LC], in which the respective assumption is used.

\begin{itemize}

\item[A\refstepcounter{assumpnum}\theassumpnum \label{A_aorta_volume}:]
The 3D image provided by the user must contains a visually distingushable aorta volume [\ddref{segmentation}].


\end{itemize}


\subsubsection{Data Definitions}\label{sec_datadef}

This section collects and defines all the data needed to build the instance
models. The dimension of each quantity is also given. 

~\newline


\noindent
\begin{minipage}{\textwidth}
\renewcommand*{\arraystretch}{1.5}
\begin{tabular}{| p{\colAwidth} | p{\colBwidth}|}
\hline
\rowcolor[gray]{0.9}
Number& DD\refstepcounter{datadefnum}\thedatadefnum \label{v}\\
\hline
Label& \bf Voxel \\
\hline
Symbol & $ \text{v} : \mathbb{R}$\\
\hline
% Units& $Mt^{-3}$\\
% \hline
  SI Units & - \\
  \hline
  Equation& - \\
  \hline
  Description & 
                A slice (\ddref{slice}) consists of n $\times$ n voxels. A real number is assigned to each voxel to reports the intensity on a grey-scale image.
  \\
  \hline
  Sources & \\
  \hline
  Ref.\ By & \ddref{slice}\\
  \hline
\end{tabular}
\end{minipage}\\ \\

% -----------------------------------------------------------------------------------------------------------------------------------------------------------------------------------------------------------------

\noindent
\begin{minipage}{\textwidth}
\renewcommand*{\arraystretch}{1.5}
\begin{tabular}{| p{\colAwidth} | p{\colBwidth}|}
\hline
\rowcolor[gray]{0.9}
Number& DD\refstepcounter{datadefnum}\thedatadefnum \label{slice}\\
\hline
Label& \bf Image/Slice \\
\hline
Symbol & $ \text{slice} : \mathbb{R}^{m \times n}$\\
\hline
% Units& $Mt^{-3}$\\
% \hline
  SI Units & - \\
  \hline
  Equation& - \\
  \hline
  Description & 
                A visual representation of something that is represented using only two spatial dimensions with a sequence of arrays where a voxel (\ddref{v}) represents the color or intensity. Each move in the Z plane is considered as one slice
  \\
  \hline
  Sources & \\
  \hline
  Ref.\ By & \ddref{volume}\\
  \hline
\end{tabular}
\end{minipage}\\ \\

% -----------------------------------------------------------------------------------------------------------------------------------------------------------------------------------------------------------------

\noindent
\begin{minipage}{\textwidth}
\renewcommand*{\arraystretch}{1.5}
\begin{tabular}{| p{\colAwidth} | p{\colBwidth}|}
\hline
\rowcolor[gray]{0.9}
Number& DD\refstepcounter{datadefnum}\thedatadefnum \label{volume}\\
\hline
Label& \bf Volume\\
\hline
Symbol & $ \text{volume} : \mathbb{R}^{m \times n \times p}$\\
\hline
% Units& $Mt^{-3}$\\
% \hline
  SI Units & - \\
  \hline
  Equation& - \\
  \hline
  Description & 
                A three dimensional image is a sequence of some images/slices (\ddref{slice}).
  \\
  \hline
  Sources & \\
  \hline
  Ref.\ By & \iref{roi}\\
  \hline
\end{tabular}
\end{minipage}\\

\subsubsection{Instance Models} \label{sec_instance}    

This section transforms the problem defined in Section~\ref{Sec_pd} into 
one which is expressed in mathematical terms. It uses concrete symbols defined 
in Section~\ref{sec_datadef} to replace the abstract symbols in the models 
identified in Sections~\ref{sec_theoretical} and~\ref{sec_gendef}. \\

\noindent 
The goals \gsref{Gsegment} are solved by finding \iref{roi} and perform \iref{segmentation} on the descending and ascending aorta.

~\newline

%Instance Model 1

\noindent
\begin{minipage}{\textwidth}
\renewcommand*{\arraystretch}{1.5}
\begin{tabular}{| p{\colAwidth} | p{\colBwidth}|}
  \hline
  \rowcolor[gray]{0.9}
  Number& IM\refstepcounter{instnum}\theinstnum \label{roi}\\
  \hline
  Label& \bf Region of interest \\
  \hline
  Inputs & $ \text{V}_\text{in}:\mathbb{R}^{m_i \times n_i \times p_i}$ , $\text{Start} : \mathbb{N}^3 $, $m_o, n_o, p_o : \mathbb{N}$,  with the following constraints:
\begin{center}
$ 0 \leq start[0] \leq (m_i-1) $ \\
$ 0 \leq start[1] \leq (n_i-1) $ \\
$ 0 \leq start[2] \leq (p_i-1) $ \\
$ 0 < m_o < (m_i-start[0]) $ \\
$ 0 < n_o < (n_i-start[1]) $ \\
$ 0 < p_o < (p_i-start[2]) $ \\
\end{center}\\
  \hline
  Output& $ \text{V}_\text{out} : \mathbb{R}^{m_o \times n_o \times p_o}$ such that
\begin{center}
$ \forall (i,j,k : \mathbb{N} | $ \\
$ i \in [start[0]..start[0]+m_o] \wedge $ \\
$ j \in [start[1]..start[1]+n_o] \wedge $ \\
$ k \in [start[2]..start[2]+p_o] :$\\
$ V_{out}[i][j][k]=V_{in}[i][j][k])$
\end{center}\\
  \hline
  Description & The regions of interest is a subset (shaped like a box) of the 3D Vout. This subset contains the anatomical structure that the users wants to read, process or extract. 
%~\newline For Starting\_voxel $[Sv_1, Sv_2, Sv_3]$ and for Size $[S_1, S_2, S_3]$:
%\begin{center}
%$  m_o=S_1  \; \wedge \;  n_o=S_2   \; \wedge \;  p_o=S_3 $
%$ \forall i,j,k:\mathbb{N} \; \{i \in [0, m_o-1]  \; \wedge \; j \in [0,n_o-1] \; \wedge \; k \in [0, p_o-1] \} $. 
%$ \text{V}_\text{out}[i,j,k] =  \text{V}_\text{in}[Sv_1+i, Sv_2+j, Sv_3+k] $. 
%\end{center}
  \\
  \hline
  Sources&  \\
  \hline
  Ref.\ By & \iref{segmentation} \\
  \hline
\end{tabular}
\end{minipage}\\

~\newline

%Instance Model 1

\noindent
\begin{minipage}{\textwidth}
\renewcommand*{\arraystretch}{1.5}
\begin{tabular}{| p{\colAwidth} | p{\colBwidth}|}
  \hline
  \rowcolor[gray]{0.9}
  Number& IM\refstepcounter{instnum}\theinstnum \label{segmentation}\\
  \hline
  Label& \bf Segmentation \\
  \hline
  Input & $ \text{V}_\text{in} : \mathbb{R}^{m \times n \times p}$ \\
  \hline
  Output& $ \text{V}_\text{out} : \mathbb{R}^{m \times n \times p}$ such that
\begin{center}
$ \forall (i,j,k : \mathbb{N} $         $|$ \\
$ i \in [0..m-1] \wedge $ \\
$ j \in [1..n-1] \wedge $ \\
$ k \in[2..p-1] :$\\
$ (V_{in}[i,j,k] \in \text{structure} \implies V_{out}[i,j,k]=HIGH /)$\\
$ V_{in}[i,j,k] \notin \text{structure} \implies V_{out}[i,j,k]=LOW)) $
\end{center}\\

  \hline
  Description & The process of extract an anotomical structure from the original 3D volume. The extracted anotomical structure is represented with high intensity pixel value. The rest of the image should have a lower intensity pixel value. A seed is what the algorithm needed as the inputs to perform segmentation, the type of a seed is different among different algorithm.
%\newline For Vout:
%\begin{center}
%$ \forall i,j,k : \mathbb{N} \{i \in [0, m-1]  \; \wedge \; j \in [0,n-1] \; \wedge \; k \in [0, p-1] \} $.
%\[
%  \text{V}_\text{out}[i,j,k] = 
%  \begin{cases} 
%   HIGH & \text{if } \text{V}_\text{in}[i,j,k] \in \text{part of the anotomical structure} \\
%   LOW  & \text{otherwise}
%  \end{cases}
%\]
%\end{center}
  \\
  \hline
  Sources& \\
  \hline
  Ref.\ By & \rref{R_output}, \lcref{LC_seg_algorithm} \\
  \hline
\end{tabular}
\end{minipage}\\

%~\newline

%\subsubsection*{Derivation of ...}


%\subsubsection{Input Data Constraints} \label{sec_DataConstraints}    
%Data constraints on the input are as follows:
%\newline
%\begin{itemize}
%\item{ For Starting\_voxel $[Sv_1, Sv_2, Sv_3]$, $ \text{V}_\text{in}:\mathbb{R}^{m_i \times n_i \times p_i}$ , and Size $[S_1, S_2, S_3]$:}\\
%$ 0\leq Sv_1\leq m_i  \; \wedge \;  0 \leq  Sv_2 \leq n_i   \; \wedge \;  0 \leq  Sv_3 \leq p_i $. \\
%$ 0 < S_1  \le m_i  \; \wedge \;  0 <  S_2 \le n_i   \; \wedge \;  0 <  S_3 \le p_i $.\\
%$ 0 \leq  Sv_1 + S_1  \leq m_i  \; \wedge \;  0 \leq  Sv_2 +S_2 \leq n_i   \; \wedge \;  0 \leq  Sv_3+S_3 \leq p_i $\\
%\end{itemize}

%Table~\ref{TblInputVar} shows the data constraints on the input output
%variables.  The column for physical constraints gives the physical limitations
%on the range of values that can be taken by the variable.  The column for
%software constraints restricts the range of inputs to reasonable values.  The%
%software constraints will be helpful in the design stage for picking suitable
%algorithms.  The constraints are conservative, to give the user of the model the
%flexibility to experiment with unusual situations.  The column of typical values
%is intended to provide a feel for a common scenario.  The uncertainty column
%provides an estimate of the confidence with which the physical quantities can be
%measured.  This information would be part of the input if one were performing an
% uncertainty quantification exercise.

% The specification parameters in Table~\ref{TblInputVar} are listed in
% Table~\ref{TblSpecParams}.

%\begin{table}[!h]
%  \caption{Input Variables} \label{TblInputVar}
%  \renewcommand{\arraystretch}{1.2}
%\noindent \begin{longtable*}{l l l l c} 
%  \toprule
%  \textbf{Var} & \textbf{Physical Constraints} & \textbf{Software Constraints} &
%                             \textbf{Typical Value} & \textbf{Uncertainty}\\
%  \midrule 
%$ 0\leq Sv_1\leq m_i  \; \wedge \;  0 \leq  Sv_2 \leq n_i   \; \wedge \;  0 \leq  Sv_3 \leq p_i $. 
%$ 0 < S_1  \le m_i  \; \wedge \;  0 <  S_2 \le n_i   \; \wedge \;  0 <  S_3 \le p_i $.
%$ 0 \leq  Sv_1 + S_1  \leq m_i  \; \wedge \;  0 \leq  Sv_2 +S_2 \leq n_i   \; \wedge \;  0 \leq  Sv_3+S_3 \leq p_i $

  % Size $\it{sequence}[0..2]$ & $x>0 \forall x \in Size$ & $L_{\text{min}} \leq L \leq L_{\text{max}}$ & 1.5 \si[per-mode=symbol] {\metre} & 10\%
%  \\
%  \bottomrule
%\end{longtable*}
%\end{table}

%\begin{table}[!h]
%\caption{Specification Parameter Values} \label{TblSpecParams}
%\renewcommand{\arraystretch}{1.2}
%\noindent \begin{longtable*}{l l} 
%  \toprule
%  \textbf{Var} & \textbf{Value} \\
%  \midrule 
%  $L_\text{min}$ & 0.1 \si{\metre}\\
%  \bottomrule
%\end{longtable*}
%\end{table}
%
%\subsubsection{Properties of a Correct Solution} \label{sec_CorrectSolution}
%
%\noindent
%A correct solution must exhibit \plt{fill in the details}.  \plt{These
%  properties are in addition to the stated requirements.  There is no need to
%  repeat the requirements here.  These additional properties may not exist for
%  every problem.  Examples include conservation laws (like conservation of
%  energy or mass) and known constraints on outputs, which are usually summarized
%  in tabular form.  A sample table is shown in Table~\ref{TblOutputVar}}

%\begin{table}[!h]
%\caption{Output Variables} \label{TblOutputVar}
%\renewcommand{\arraystretch}{1.2}
%\noindent \begin{longtable*}{l l} 
%  \toprule
%  \textbf{Var} & \textbf{Physical Constraints} \\
%  \midrule 
%  $T_W$ & $T_\text{init} \leq T_W \leq T_C$ (by~\aref{A_charge})
%  \\
%  \bottomrule
%\end{longtable*}
%\end{table}
%
%\plt{This section is not for test cases or techniques for verification and
%  validation.  Those topics will be addressed in the Verification and Validation
%  plan.}

\section{Requirements}

This section provides the functional requirements, the business tasks that the
software is expected to complete, and the nonfunctional requirements, the
qualities that the software is expected to exhibit.

\subsection{Functional Requirements}

\noindent \begin{itemize}
\item[R\refstepcounter{reqnum}\thereqnum \label{R_Inputs}:] Input the following
  functions, data and parameters:
  \renewcommand{\arraystretch}{1.2}
%\noindent \begin{tabularx}{1.0\textwidth}{l l X}
  \noindent \begin{longtable*}{l l p{12cm}} \toprule
         \textbf{symbol}  & \textbf{description}\\
         \midrule 
$volume$ & CT Scans volume
(\ddref{volume})\\ %comparing value used for excluding correlation coefficients
$Seed$ & Any inputs neede by the segmentation algorithm (\iref{segmentation})\\
         \bottomrule
\end{longtable*}

\item[R\refstepcounter{reqnum}\thereqnum \label{R_roi}:] Use the volume in \rref{R_Inputs} to create a second volume, the region of interest (\iref{roi}) that contains all of the aorta.

\item[R\refstepcounter{reqnum}\thereqnum \label{R_output}:] Perform segmentation (\iref{segmentation}) on the volume created in \rref{R_roi}.

\item[R\refstepcounter{reqnum}\thereqnum \label{R_visualize}:] Visualize a volume (\ddref{volume}).

 %non functional \item[R\refstepcounter{reqnum}\thereqnum \label{R_seed}:] Allow user to input various data type of data for seed.

% \item[R\refstepcounter{reqnum}\thereqnum \label{R_output}:] Allow user to link to segmentation library, and perform segmentation (\iref{segmentation}) on the input data.
\end{itemize}

\subsection{Nonfunctional Requirements}

\noindent \begin{itemize}

%\item[NFR\refstepcounter{nfrnum}\thenfrnum \label{NFR_Accuracy}:]
%  \textbf{Accuracy} \plt{Characterize the accuracy by giving the context/use for
%    the software.  Maybe something like, ``The accuracy of the computed
%    solutions should meet the level needed for $<$engineering or scientific
%    application$>$.  The level of accuracy achieved by \progname{} shall be
%    described following the procedure given in Section~X of the Verification and
%    Validation Plan.''  A link to the VnV plan would be a nice extra.}

\item[NFR\refstepcounter{nfrnum}\thenfrnum \label{NFR_GUI}:] \textbf{Usability}
\progname{} provides a user-friendly interface to import any DICOM files, and input the required parameters.

\item[NFR\refstepcounter{nfrnum}\thenfrnum \label{NFR_visualize}:] \textbf{Usability}
\progname{} can visualize the volume with 3D rendering.


%\item[NFR\refstepcounter{nfrnum}\thenfrnum \label{NFR_Maintainability}:]
%  \textbf{Maintainability} \plt{The effort required to make any of the likely
%    changes listed for \progname{} should be less than FRACTION of the original
%    development time.  FRACTION is then a symbolic constant that can be defined
%    at the end of the report.}
%
%\item[NFR\refstepcounter{nfrnum}\thenfrnum \label{NFR_Portability}:]
%  \textbf{Portability} \plt{This NFR is easier to write than the others.  The
%    systems that \progname{} should run on should be listed here.  When possible
%    the specific versions of the potential operating environments should be
%    given.  To make the NFR verifiable a statement could be made that the tests
%    from a given section of the VnV plan can be successfully run on all of the
%    possible operating environments.}

\item Other NFRs that might be discussed include verifiability,
  understandability and reusability.

\end{itemize}

\section{Likely Changes}    

\noindent \begin{itemize}

\item[LC\refstepcounter{lcnum}\thelcnum\label{LC_seg_algorithm}:] \iref{segmentation} There are various segmentation algorithms, each has a different procedure and inputs.
	
\end{itemize}


\section{Traceability Matrices and Graphs}

The purpose of the traceability matrices is to provide easy references on what
has to be additionally modified if a certain component is changed.  Every time a
component is changed, the items in the column of that component that are marked
with an ``X'' may have to be modified as well.  Table~\ref{Table:trace} shows the
dependencies of theoretical models, general definitions, data definitions, and
instance models with each other. Table~\ref{Table:R_trace} shows the
dependencies of instance models, requirements, and data constraints on each
other. Table~\ref{Table:A_trace} shows the dependencies of theoretical models,
general definitions, data definitions, instance models, and likely changes on
the assumptions.

\plt{You will have to modify these tables for your problem.}

\plt{The traceability matrix is not generally symmetric.  If GD1 uses A1, that
  means that GD1's derivation or presentation requires invocation of A1.  A1
  does not use GD1.  A1 is ``used by'' GD1.}

\plt{The traceability matrix is challenging to maintain manually.  Please do
  your best.  In the future tools (like Drasil) will make this much easier.}

\afterpage{
\begin{landscape}
\begin{table}[h!]
\centering
\begin{tabular}{|c|c|c|c|c|c|c|c|c|c|c|c|c|c|c|c|c|c|c|c|}
\hline
	& \aref{A_OnlyThermalEnergy}& \aref{A_hcoeff}& \aref{A_mixed}& \aref{A_tpcm}& \aref{A_const_density}& \aref{A_const_C}& \aref{A_Newt_coil}& \aref{A_tcoil}& \aref{A_tlcoil}& \aref{A_Newt_pcm}& \aref{A_charge}& \aref{A_InitTemp}& \aref{A_OpRangePCM}& \aref{A_OpRange}& \aref{A_htank}& \aref{A_int_heat}& \aref{A_vpcm}& \aref{A_PCM_state}& \aref{A_Pressure} \\
\hline
\tref{T_COE}        & X& & & & & & & & & & & & & & & & & & \\ \hline
\tref{T_SHE}        & & & & & & & & & & & & & & & & & & & \\ \hline
\tref{T_LHE}        & & & & & & & & & & & & & & & & & & & \\ \hline
\dref{NL}           & & X& & & & & & & & & & & & & & & & & \\ \hline
\dref{ROCT}         & & & X& X& X& X& & & & & & & & & & & & & \\ \hline
\ddref{FluxCoil}    & & & & & & & X& X& X& & & & & & & & & & \\ \hline
\ddref{FluxPCM}     & & & X& X& & & & & & X& & & & & & & & & \\ \hline
\ddref{D_HOF}       & & & & & & & & & & & & & & & & & & & \\ \hline
\ddref{D_MF}        & & & & & & & & & & & & & & & & & & & \\ \hline
\iref{ewat}         & & & & & & & & & & & X& X& & X& X& X& & & X \\ \hline
\iref{epcm}         & & & & & & & & & & & & X& X& & & X& X& X& \\ \hline
\iref{I_HWAT}       & & & & & & & & & & & & & & X& & & & & X \\ \hline
\iref{I_HPCM}       & & & & & & & & & & & & & X& & & & & X & \\ \hline
\lcref{LC_tpcm}     & & & & X& & & & & & & & & & & & & & & \\ \hline
\lcref{LC_tcoil}    & & & & & & & & X& & & & & & & & & & & \\ \hline
\lcref{LC_tlcoil}   & & & & & & & & & X& & & & & & & & & & \\ \hline
\lcref{LC_charge}   & & & & & & & & & & & X& & & & & & & & \\ \hline
\lcref{LC_InitTemp} & & & & & & & & & & & & X& & & & & & & \\ \hline
\lcref{LC_htank}    & & & & & & & & & & & & & & & X& & & & \\
\hline
\end{tabular}
\caption{Traceability Matrix Showing the Connections Between Assumptions and Other Items}
\label{Table:A_trace}
\end{table}
\end{landscape}
}

\begin{table}[h!]
\centering
\begin{tabular}{|c|c|c|c|c|c|c|c|c|c|c|c|c|c|c|c|c|c|c|c|c|c|c|c|}
\hline        
	& \tref{T_COE}& \tref{T_SHE}& \tref{T_LHE}& \dref{NL}& \dref{ROCT} & \ddref{FluxCoil}& \ddref{FluxPCM} & \ddref{D_HOF}& \ddref{D_MF}& \iref{ewat}& \iref{epcm}& \iref{I_HWAT}& \iref{I_HPCM} \\
\hline
\tref{T_COE}     & & & & & & & & & & & & & \\ \hline
\tref{T_SHE}     & & & X& & & & & & & & & & \\ \hline
\tref{T_LHE}     & & & & & & & & & & & & & \\ \hline
\dref{NL}        & & & & & & & & & & & & & \\ \hline
\dref{ROCT}      & X& & & & & & & & & & & & \\ \hline
\ddref{FluxCoil} & & & & X& & & & & & & & & \\ \hline
\ddref{FluxPCM}  & & & & X& & & & & & & & & \\ \hline
\ddref{D_HOF}    & & & & & & & & & & & & & \\ \hline
\ddref{D_MF}     & & & & & & & & X& & & & & \\ \hline
\iref{ewat}      & & & & & X& X& X& & & & X& & \\ \hline
\iref{epcm}      & & & & & X& & X& & X& X& & & X \\ \hline
\iref{I_HWAT}    & & X& & & & & & & & & & & \\ \hline
\iref{I_HPCM}    & & X& X& & & & X& X& X& & X& & \\
\hline
\end{tabular}
\caption{Traceability Matrix Showing the Connections Between Items of Different Sections}
\label{Table:trace}
\end{table}

\begin{table}[h!]
\centering
\begin{tabular}{|c|c|c|c|c|c|c|c|}
\hline
	& \iref{ewat}& \iref{epcm}& \iref{I_HWAT}& \iref{I_HPCM}& \ref{sec_DataConstraints}& \rref{R_RawInputs}& \rref{R_MassInputs} \\
\hline
\iref{ewat}            & & X& & & & X& X \\ \hline
\iref{epcm}            & X& & & X& & X& X \\ \hline
\iref{I_HWAT}          & & & & & & X& X \\ \hline
\iref{I_HPCM}          & & X& & & & X& X \\ \hline
\rref{R_RawInputs}     & & & & & & & \\ \hline
\rref{R_MassInputs}    & & & & & & X& \\ \hline
\rref{R_CheckInputs}   & & & & & X& & \\ \hline
\rref{R_OutputInputs}  & X& X& & & & X& X \\ \hline
\rref{R_TempWater}     & X& & & & & & \\ \hline 
\rref{R_TempPCM}       & & X& & & & & \\ \hline
\rref{R_EnergyWater}   & & & X& & & & \\ \hline
\rref{R_EnergyPCM}     & & & & X& & & \\ \hline
\rref{R_VerifyOutput}  & & & X& X& & & \\ \hline
\rref{R_timeMeltBegin} & & X& & & & & \\ \hline
\rref{R_timeMeltEnd}   & & X& & & & & \\ 
\hline
\end{tabular}
\caption{Traceability Matrix Showing the Connections Between Requirements and Instance Models}
\label{Table:R_trace}
\end{table}

The purpose of the traceability graphs is also to provide easy references on
what has to be additionally modified if a certain component is changed.  The
arrows in the graphs represent dependencies. The component at the tail of an
arrow is depended on by the component at the head of that arrow. Therefore, if a
component is changed, the components that it points to should also be
changed. Figure~\ref{Fig_ATrace} shows the dependencies of theoretical models,
general definitions, data definitions, instance models, likely changes, and
assumptions on each other. Figure~\ref{Fig_RTrace} shows the dependencies of
instance models, requirements, and data constraints on each other.

% \begin{figure}[h!]
% 	\begin{center}
% 		%\rotatebox{-90}
% 		{
% 			\includegraphics[width=\textwidth]{ATrace.png}
% 		}
% 		\caption{\label{Fig_ATrace} Traceability Matrix Showing the Connections Between Items of Different Sections}
% 	\end{center}
% \end{figure}


% \begin{figure}[h!]
% 	\begin{center}
% 		%\rotatebox{-90}
% 		{
% 			\includegraphics[width=0.7\textwidth]{RTrace.png}
% 		}
% 		\caption{\label{Fig_RTrace} Traceability Matrix Showing the Connections Between Requirements, Instance Models, and Data Constraints}
% 	\end{center}
% \end{figure}

\section{Development Plan}

\plt{This section is optional.  It is used to explain the plan for developing
  the software.  In particular, this section gives a list of the order in which
  the requirements will be implemented.  In the context of a course  this is
  where you can indicate which requirements will be implemented as part of the
  course, and which will be ``faked'' as future work.  This section can be
  organized as a prioritized list of requirements, or it could should the
  requirements that will be implemented for ``phase 1'', ``phase 2'', etc.}

\section{Values of Auxiliary Constants}

\plt{Show the values of the symbolic parameters introduced in the report.}

\plt{The definition of the requirements will likely call for SYMBOLIC\_CONSTANTS.
Their values are defined in this section for easy maintenance.}

\plt{The value of FRACTION, for the Maintainability NFR would be given here.}

\newpage

\bibliographystyle {plainnat}
\bibliography {../../refs/References}

\newpage

\noindent \plt{The following is not part of the template, just some things to consider
  when filing in the template.}

\noindent \plt{Grammar, flow and \LaTeX advice:
\begin{itemize}
\item For Mac users \texttt{*.DS\_Store} should be in \texttt{.gitignore}
\item \LaTeX{} and formatting rules
\begin{itemize}
\item Variables are italic, everything else not, includes subscripts (link to
  document)
\begin{itemize}
\item \href{https://physics.nist.gov/cuu/pdf/typefaces.pdf}{Conventions}
\item Watch out for implied multiplication
\end{itemize}
\item Use BibTeX
\item Use cross-referencing
\end{itemize}
\item Grammar and writing rules
\begin{itemize}
\item Acronyms expanded on first usage (not just in table of acronyms)
\item ``In order to'' should be ``to''
\end{itemize}
\end{itemize}}

\noindent \plt{Advice on using the template:
\begin{itemize}
\item Difference between physical and software constraints
\item Properties of a correct solution means \emph{additional} properties, not
  a restating of the requirements (may be ``not applicable'' for your problem).
  If you have a table of output constraints, then these are properties of a
  correct solution.
\item Assumptions have to be invoked somewhere
\item ``Referenced by'' implies that there is an explicit reference
\item Think of traceability matrix, list of assumption invocations and list of
  reference by fields as automatically generatable
\item If you say the format of the output (plot, table etc), then your
  requirement could be more abstract
\end{itemize}
}

\end{document}