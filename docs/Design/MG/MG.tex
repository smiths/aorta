\documentclass[12pt, titlepage]{article}

\usepackage{fullpage}
\usepackage[round]{natbib}
\usepackage{multirow}
\usepackage{booktabs}
\usepackage{tabularx}
\usepackage{graphicx}
\usepackage{float}
\usepackage{hyperref}
\hypersetup{
    colorlinks,
    citecolor=blue,
    filecolor=black,
    linkcolor=red,
    urlcolor=blue
}

\input{../../Comments}
%% Common Parts

\newcommand{\progname}{AortaGeomRecon} % PUT YOUR PROGRAM NAME HERE
\newcommand{\authname}{Jingyi Lin} % AUTHOR NAMES                  

\usepackage{hyperref}
    \hypersetup{colorlinks=true, linkcolor=blue, citecolor=blue, filecolor=blue,
                urlcolor=blue, unicode=false}
    \urlstyle{same}
                                


\newcounter{acnum}
\newcommand{\actheacnum}{AC\theacnum}
\newcommand{\acref}[1]{AC\ref{#1}}

\newcounter{ucnum}
\newcommand{\uctheucnum}{UC\theucnum}
\newcommand{\uref}[1]{UC\ref{#1}}

\newcounter{mnum}
\newcommand{\mthemnum}{M\themnum}
\newcommand{\mref}[1]{M\ref{#1}}

\begin{document}

\title{Module Guide for \progname{}} 
\author{\authname}
\date{\today}

\maketitle

\pagenumbering{roman}

\section{Revision History}

\begin{tabularx}{\textwidth}{p{3cm}p{2cm}X}
\toprule {\bf Date} & {\bf Version} & {\bf Notes}\\
\midrule
2022-10-18 & 1.0 & First draft of Module Guide\\
2023-04-21 & 1.1 & Second draft of Module Guide\\
\bottomrule
\end{tabularx}

\newpage

\section{Reference Material}

This section records information for easy reference.

\subsection{Abbreviations and Acronyms}

\renewcommand{\arraystretch}{1.2}
\begin{tabular}{l l} 
  \toprule		
  \textbf{Symbol} & \textbf{description}\\
  \midrule 
  AC & Anticipated Change\\
  DAG & Directed Acyclic Graph \\
  M & Module \\
  MG & Module Guide \\
  OS & Operating System \\
  R & Requirement\\
  NFR & Non-functional requirements\\
  SC & Scientific Computing \\
  SRS & Software Requirements Specification\\
  \progname & Aorta geometry reconstruction\\
  UC & Unlikely Change \\
  \bottomrule
\end{tabular}\\

\newpage

\tableofcontents

\listoftables

\listoffigures

\newpage

\pagenumbering{arabic}

\section{Introduction}

Decomposing a system into modules is a commonly accepted approach to developing
software.  A module is a work assignment for a programmer or programming
team~\citep{ParnasEtAl1984}.  We advocate a decomposition
based on the principle of information hiding~\citep{Parnas1972a}.  This
principle supports design for change, because the ``secrets'' that each module
hides represent likely future changes.  Design for change is valuable in SC,
where modifications are frequent, especially during initial development as the
solution space is explored.  

Our design follows the rules laid out by \citet{ParnasEtAl1984}, as follows:
\begin{itemize}
\item System details that are likely to change independently should be the
  secrets of separate modules.
\item Each data structure is implemented in only one module.
\item Any other program that requires information stored in a module's data
  structures must obtain it by calling access programs belonging to that module.
\end{itemize}

After completing the first stage of the design, the Software Requirements
Specification (SRS), the Module Guide (MG) is developed~\citep{ParnasEtAl1984}. The MG
specifies the modular structure of the system and is intended to allow both
designers and maintainers to easily identify the parts of the software.  The
potential readers of this document are as follows:

\begin{itemize}
\item New project members: This document can be a guide for a new project member
  to easily understand the overall structure and quickly find the
  relevant modules they are searching for.
\item Maintainers: The hierarchical structure of the module guide improves the
  maintainers' understanding when they need to make changes to the system. It is
  important for a maintainer to update the relevant sections of the document
  after changes have been made.
\item Designers: Once the module guide has been written, it can be used to
  check for consistency, feasibility and flexibility. Designers can verify the
  system in various ways, such as consistency among modules, feasibility of the
  decomposition, and flexibility of the design.
\end{itemize}

The rest of the document is organized as follows. Section
\ref{SecChange} lists the anticipated and unlikely changes of the software
requirements. Section \ref{SecMH} summarizes the module decomposition that
was constructed according to the likely changes. Section \ref{SecConnection}
specifies the connections between the software requirements and the
modules. Section \ref{SecMD} gives a detailed description of the
modules. Section \ref{SecTM} includes two traceability matrices. One checks
the completeness of the design against the requirements provided in the SRS. The
other shows the relation between anticipated changes and the modules. Section
\ref{SecUse} describes the use relation between modules.

\section{Anticipated and Unlikely Changes} \label{SecChange}

This section lists possible changes to the system. According to the likeliness
of the change, the possible changes are classified into two
categories. Anticipated changes are listed in Section \ref{SecAchange}, and
unlikely changes are listed in Section \ref{SecUchange}.

\subsection{Anticipated Changes} \label{SecAchange}

Anticipated changes are the source of the information that is to be hidden
inside the modules. Ideally, changing one of the anticipated changes will only
require changing the one module that hides the associated decision. The approach
adapted here is called design for
change.

\begin{description}
\item[\refstepcounter{acnum} \actheacnum \label{acHardware}:] The specific hardware on which the software is running.
\item[\refstepcounter{acnum} \actheacnum \label{acInput}:] The format of the initial input data.
\item[\refstepcounter{acnum} \actheacnum \label{acAlgo}:] The algorithm to segment the aorta.
\item[\refstepcounter{acnum} \actheacnum \label{acInputParams}:] The input parameters required to execute the algorithm.
\item[\refstepcounter{acnum} \actheacnum \label{acInterface}:] The methods to create an user interface.
\item[\refstepcounter{acnum} \actheacnum \label{acGetROI}:] The methods to retrieve a region of interest.
\item[\refstepcounter{acnum} \actheacnum \label{acVisualize}:] The methods to visualize a volume.
\item[\refstepcounter{acnum} \actheacnum \label{acControl}:] The program state is likely changing during the processing.
\item[\refstepcounter{acnum} \actheacnum \label{acOutput}:] The output data varies in data type.

\item ...
\end{description}

\subsection{Unlikely Changes} \label{SecUchange}

The module design should be as general as possible. However, a general system is
more complex. Sometimes this complexity is not necessary. Fixing some design
decisions at the system architecture stage can simplify the software design. If
these decision should later need to be changed, then many parts of the design
will potentially need to be modified. Hence, it is not intended that these
decisions will be changed.

\begin{description}
\item[\refstepcounter{ucnum} \uctheucnum \label{ucIO}:] Input/Output devices
  (Input: File and/or Keyboard, Output: File, Memory, and/or Screen).
\item[\refstepcounter{ucnum} \uctheucnum \label{ucVolDim}:] The input volume data's dimension is unlikely to change.
\item ...
\end{description}

\section{Module Hierarchy} \label{SecMH}

This section provides an overview of the module design. Modules are summarized
in a hierarchy decomposed by secrets in Table \ref{TblMH}. The modules listed
below, which are leaves in the hierarchy tree, are the modules that will
actually be implemented.

\begin{description}
\item [\refstepcounter{mnum} \mthemnum \label{mHH}:] Hardware-Hiding Module
\item [\refstepcounter{mnum} \mthemnum \label{mInput}:] Input Format Module
\item [\refstepcounter{mnum} \mthemnum \label{mParams}:] Input Parameter Module
\item [\refstepcounter{mnum} \mthemnum \label{mControl}:] Control Module
\item [\refstepcounter{mnum} \mthemnum \label{mGUI}:] GUI Module
\item [\refstepcounter{mnum} \mthemnum \label{mDisplay}:] Display Module
\item [\refstepcounter{mnum} \mthemnum \label{mROI}:] Crop Volume Module
\item [\refstepcounter{mnum} \mthemnum \label{mAortaSeg}:] Aorta Segmentation Module
\item [\refstepcounter{mnum} \mthemnum \label{mAS}:] Axial Segmentation Module
%\item [\refstepcounter{mnum} \mthemnum \label{mSS}:] Sagittal Segmentation Module
\item [\refstepcounter{mnum} \mthemnum \label{mSITK}:] SITK Module
\item [\refstepcounter{mnum} \mthemnum \label{mNP}:] NumPy Module
\item ...
\end{description}


\begin{table}[h!]
\centering
\begin{tabular}{p{0.3\textwidth} p{0.4\textwidth}p{0.3\textwidth}}
\toprule
\textbf{Level 1} & \textbf{Level 2} & \textbf{Level 3}\\
\midrule

{Hardware-Hiding Module} & ~ \\
\midrule

\multirow{7}{0.3\textwidth}{Behaviour-Hiding Module} 
&  Input Format Module & \\
&  Input Parameter Module & \\
& Control Module  & \\
& Display Module  & \\
& Crop Module & \\
& Aorta Segmentation Module & \\
\midrule

\multirow{3}{0.3\textwidth}{Software Decision Module}
& GUI Module & \\
& SITK Module \\
& Numpy Module\\
& Axial Aorta Segmentation Module\\
\bottomrule

\end{tabular}
\caption{Module Hierarchy}
\label{TblMH}
\end{table}

\section{Connection Between Requirements and Design} \label{SecConnection}

The design of the system is intended to satisfy the requirements developed in
the SRS. In this stage, the system is decomposed into modules. The connection
between requirements and modules is listed in Table \ref{TblRT}.

\section{Module Decomposition} \label{SecMD}

Modules are decomposed according to the principle of ``information hiding''
proposed by \citet{ParnasEtAl1984}. The \emph{Secrets} field in a module
decomposition is a brief statement of the design decision hidden by the
module. The \emph{Services} field specifies \emph{what} the module will do
without documenting \emph{how} to do it. For each module, a suggestion for the
implementing software is given under the \emph{Implemented By} title. If the
entry is \emph{OS}, this means that the module is provided by the operating
system or by standard programming language libraries.  \emph{\progname{}} means the
module will be implemented by the \progname{} software.

Only the leaf modules in the hierarchy have to be implemented. If a dash
(\emph{--}) is shown, this means that the module is not a leaf and will not have
to be implemented.

\subsection{Hardware Hiding Modules (\mref{mHH})}

\begin{description}
\item[Secrets:]The data structure and algorithm used to implement the virtual
  hardware.
\item[Services:]Serves as a virtual hardware used by the rest of the
  system. This module provides the interface between the hardware and the
  software. So, the system can use it to display outputs or to accept inputs.
\item[Implemented By:] OS
\end{description}

\subsection{Behaviour-Hiding Module}

\begin{description}
\item[Secrets:]The contents of the required behaviors.
\item[Services:]Includes programs that provide externally visible behavior of
  the system as specified in the software requirements specification (SRS)
  documents. This module serves as a communication layer between the
  hardware-hiding module and the software decision module. The programs in this
  module will need to change if there are changes in the SRS.
\item[Implemented By:] --
\end{description}

\subsubsection{Input Format Module (\mref{mInput})}
\begin{description}
\item[Secrets:]The format and structure of the input data.
\item[Services:]Converts the input data into the data structure used by the input parameters module.
\item[Implemented By:] \progname{}
\item[Source:] \href{https://joviel25.github.io/AortaGR-design-document/AortaGeomReconDisplayModule.html#AortaGeomReconDisplayModule.AortaGeomReconDisplayModuleLogic.process}{AortaGeomReconDisplayModuleLogic module's process function}
%\item[Type of Module:] [Abstract Data Type]
%  [Information to include for leaf modules in the decomposition by secrets tree.]
\end{description}

\subsubsection{Input Parameter Module (\mref{mParams})}

\begin{description}
\item[Secrets:] The data structure for input parameters, how the
values are input and how the values are verified.  The load and verify secrets
are isolated to their own access programs.
\item[Services:] Gets input from user, stores input and verifies that the
  input parameters comply with physical and software constraints. Throws an
  error if a parameter violates a physical constraint. Throws a warning if a
  parameter violates a software constraint.  Stored parameters can be read
  individually, but write access is only to redefine the entire set of inputs.
 % This module knows how many parameters it stores.
\item[Implemented By:] \progname{}
\item[Source:] \href{https://joviel25.github.io/AortaGR-design-document/AortaGeomReconDisplayModuleLib.html#AortaSegmenter.AortaSegmenter}{AortaSegmenter class' atributes}
\end{description}

\subsubsection{Control Module (\mref{mControl})}

\begin{description}
\item[Secrets:]The algorithm for coordinating the running of the program.
\item[Services:]Provides the main program's entry point, the ability to jump from a program state to another.
\item[Implemented By:] \progname
\item[Source:] \href{https://joviel25.github.io/AortaGR-design-document/AortaGeomReconDisplayModule.html#AortaGeomReconDisplayModule.AortaGeomReconDisplayModuleWidget.onApplyButton}{AortaGeomReconDisplayModuleWidget module}
\end{description}

\subsubsection{Display Module (\mref{mDisplay})}
\begin{description}
\item[Secrets:]The methods by which it displays input data and output data, along 
with any other info to the screen.
\item[Services:]Display the aorta images and vtk 3D geometry.
\item[Implemented By:] 3D Slicer
\end{description}

\subsubsection{Crop Module (\mref{mROI})}
\begin{description}
\item[Secrets:] The parameters, libraries to retrieve a region of interest from a volume.s
\item[Services:] Import the necessary libraries, store the input parameter, and coordinate the uses of these data and libraries to retrieve a region of interest.
\item[Implemented By:] 3D Slicer
% \item[Use case:] This module should always have a \emph{begin\_segmentation} API available for the developer to begin the aorta segmentation.
\end{description}

\subsubsection{Aorta Segmentation Module (\mref{mAortaSeg})}
\begin{description}
\item[Secrets:] The parameters, libraries to perform segmentation.
\item[Services:] Import the necessary libraries, store the input parameter, and coordinate the uses of these data and libraries to perform segmentation.
\item[Implemented By:] \progname
\item[Source:] \href{https://joviel25.github.io/AortaGR-design-document/AortaGeomReconDisplayModuleLib.html#module-AortaSegmenter}{AortaSegmenter module}
\end{description}


\subsubsection{Etc.}


\subsection{Software Decision Module}

\begin{description}
\item[Secrets:] The design decision based on mathematical theorems, physical
  facts, or programming considerations. The secrets of this module are
  \emph{not} described in the SRS.
\item[Services:] Includes data structure and algorithms used in the system that
  do not provide direct interaction with the user. 
  % Changes in these modules are more likely to be motivated by a desire to
  % improve performance than by externally imposed changes.
\item[Implemented By:] --
\end{description}


\subsubsection{GUI Module (\mref{mGUI})}
\begin{description}
\item[Secrets:]The necessary library/framework to build a GUI software.
\item[Services:]Povide GUI software for user to write/read inputs, and send commands to the program. It could include the rendering a windows, inputs, keyboard and mouse interaction with the GUI elements.
\item[Implemented By:] 3D Slicer
\end{description}


\subsubsection{Simple ITK module (\mref{mSITK})}
\begin{description}
\item[Secrets:] The libraries and the APIs to perform image analysis, and image segmentation.
\item[Services:] Provides useful APIs such as ThresholdSegmentationLevelSetsImageFilter, LabelStatisticImageFilter to perform aorta segmentation.
\item[Implemented By:] SITK
% \item[Use case:] This module should always have a \emph{begin\_segmentation} API available for the developer to begin the aorta segmentation.
\end{description}

\subsubsection{NumPy module (\mref{mNP})}
\begin{description}
\item[Secrets:] The libraries and the APIs to perform element-wise mathematic operations on multi-dimensional array.
\item[Services:] Provides useful APIs such as calculating the max, min, average of multi-dimensional array. NumPy.where function provides the search in the multi-dimensional array functionality which will return any element's indexes if it satisfies the given condition.
\item[Implemented By:] NumPy
% \item[Use case:] This module should always have a \emph{begin\_segmentation} API available for the developer to begin the aorta segmentation.
\end{description}


%\item [\refstepcounter{mnum} \mthemnum \label{mSITK}:] SITK Module
%\item [\refstepcounter{mnum} \mthemnum \label{mNP}:] NumPy Module

\subsubsection{Axial Segmentation Module (\mref{mAS})}
\begin{description}
\item[Secrets:] The parameters, libraries to perform segmentation along the axial axis.
\item[Services:] Import the necessary libraries, store the input parameter, and coordinate the uses of these data and libraries to perform axial segmentation.
\item[Implemented By:] \progname
% \item[Use case:] This module should always have a \emph{begin\_segmentation} API available for the developer to begin the aorta segmentation.
\end{description}

%\subsubsection{Sagittal Segmentation Module (\mref{mSS})}
%\begin{description}
%\item[Secrets:] The parameters, libraries to perform segmentation along the Sagittal axis.
%\item[Services:] Import the necessary libraries, store the input parameter, and coordinate the uses of these data and libraries to perform sagittal segmentation.
%\item[Implemented By:] \progname
%% \item[Use case:] This module should always have a \emph{begin\_segmentation} API available for the developer to begin the aorta segmentation.
%\end{description}

\section{Traceability Matrix} \label{SecTM}

% \item [\refstepcounter{mnum} \mthemnum \label{mSAS}] Aorta Sagital Segmentation Module

This section shows two traceability matrices: between the modules and the
requirements and between the modules and the anticipated changes.

% the table should use mref, the requirements should be named, use something
% like fref
\begin{table}[H]
\centering
\begin{tabular}{p{0.2\textwidth} p{0.6\textwidth}}
\toprule
\textbf{Req.} & \textbf{Modules}\\
\midrule
R1 & \mref{mHH}, \mref{mInput}, \mref{mParams}, \mref{mControl}\\
R2 & \mref{mParams}, \mref{mROI}\\
R3 &  \mref{mAortaSeg}, \mref{mAS}, \mref{mSITK}, \mref{mNP}\\
R4 & \mref{mDisplay}\\
NFR1 & \mref{mControl}, \mref{mParams}, \mref{mGUI}\\
NFR2 & \mref{mDisplay}\\
\bottomrule
\end{tabular}
\caption{Trace Between Requirements and Modules}
\label{TblRT}
\end{table}

\begin{table}[H]
\centering
\begin{tabular}{p{0.2\textwidth} p{0.6\textwidth}}
\toprule
\textbf{AC} & \textbf{Modules}\\
\midrule
\acref{acHardware} & \mref{mHH}\\
\acref{acInput} & \mref{mInput}\\
\acref{acAlgo} & \mref{mAortaSeg}, \mref{mAS}\\
\acref{acInputParams} & \mref{mParams}\\
\acref{acInterface} & \mref{mGUI}\\
\acref{acGetROI} & \mref{mROI} \\
\acref{acVisualize} & \mref{mDisplay}\\
\acref{acControl} & \mref{mControl}\\
\acref{acOutput} & \mref{mSITK}, \mref{mNP} \\
\bottomrule
\end{tabular}
\caption{Trace Between Anticipated Changes and Modules}
\label{TblACT}
\end{table}

\section{Use Hierarchy Between Modules} \label{SecUse}

In this section, the uses' hierarchy between modules is
provided. \citet{Parnas1978} said of two programs A and B that A {\em uses} B if
correct execution of B may be necessary for A to complete the task described in
its specification. That is, A {\em uses} B if there exist situations in which
the correct functioning of A depends upon the availability of a correct
implementation of B.  Figure \ref{FigUH} illustrates the use relation between
the modules. It can be seen that the graph is a directed acyclic graph
(DAG). Each level of the hierarchy offers a testable and usable subset of the
system, and modules in the higher level of the hierarchy are essentially simpler
because they use modules from the lower levels.

%\begin{figure}[H]
%\centering
%%\includegraphics[width=0.7\textwidth]{UsesHierarchy.png}
%\caption{Use hierarchy among modules}
%\label{FigUH}
%\end{figure}

%\section*{References}

\bibliographystyle {plainnat}
\bibliography{../../../refs/References}

\end{document}