%% Generated by Sphinx.
\def\sphinxdocclass{report}
\documentclass[letterpaper,10pt,english]{sphinxmanual}
\ifdefined\pdfpxdimen
   \let\sphinxpxdimen\pdfpxdimen\else\newdimen\sphinxpxdimen
\fi \sphinxpxdimen=.75bp\relax
\ifdefined\pdfimageresolution
    \pdfimageresolution= \numexpr \dimexpr1in\relax/\sphinxpxdimen\relax
\fi
%% let collapsible pdf bookmarks panel have high depth per default
\PassOptionsToPackage{bookmarksdepth=5}{hyperref}

\PassOptionsToPackage{warn}{textcomp}
\usepackage[utf8]{inputenc}
\ifdefined\DeclareUnicodeCharacter
% support both utf8 and utf8x syntaxes
  \ifdefined\DeclareUnicodeCharacterAsOptional
    \def\sphinxDUC#1{\DeclareUnicodeCharacter{"#1}}
  \else
    \let\sphinxDUC\DeclareUnicodeCharacter
  \fi
  \sphinxDUC{00A0}{\nobreakspace}
  \sphinxDUC{2500}{\sphinxunichar{2500}}
  \sphinxDUC{2502}{\sphinxunichar{2502}}
  \sphinxDUC{2514}{\sphinxunichar{2514}}
  \sphinxDUC{251C}{\sphinxunichar{251C}}
  \sphinxDUC{2572}{\textbackslash}
\fi
\usepackage{cmap}
\usepackage[T1]{fontenc}
\usepackage{amsmath,amssymb,amstext}
\usepackage{babel}



\usepackage{tgtermes}
\usepackage{tgheros}
\renewcommand{\ttdefault}{txtt}



\usepackage[Bjarne]{fncychap}
\usepackage{sphinx}

\fvset{fontsize=auto}
\usepackage{geometry}


% Include hyperref last.
\usepackage{hyperref}
% Fix anchor placement for figures with captions.
\usepackage{hypcap}% it must be loaded after hyperref.
% Set up styles of URL: it should be placed after hyperref.
\urlstyle{same}

\addto\captionsenglish{\renewcommand{\contentsname}{Contents:}}

\usepackage{sphinxmessages}
\setcounter{tocdepth}{1}



\title{AortaGeomRecon}
\date{Mar 06, 2023}
\release{}
\author{Jingyi Lin}
\newcommand{\sphinxlogo}{\vbox{}}
\renewcommand{\releasename}{}
\makeindex
\begin{document}

\pagestyle{empty}
\sphinxmaketitle
\pagestyle{plain}
\sphinxtableofcontents
\pagestyle{normal}
\phantomsection\label{\detokenize{index::doc}}


\sphinxAtStartPar
This is the design document for AortaGeomRecon module, a {\hyperref[\detokenize{glossary:term-3D-Slicer}]{\sphinxtermref{\DUrole{xref,std,std-term}{3D Slicer}}}} extension to perform {\hyperref[\detokenize{glossary:term-Aorta}]{\sphinxtermref{\DUrole{xref,std,std-term}{Aorta}}}} segmentation and Aorta geometry reconstruction.


\chapter{AortaGeomReconModule}
\label{\detokenize{modules:aortageomreconmodule}}\label{\detokenize{modules::doc}}

\section{test package}
\label{\detokenize{test:test-package}}\label{\detokenize{test::doc}}

\subsection{Submodules}
\label{\detokenize{test:submodules}}

\subsection{conftest module}
\label{\detokenize{test:module-conftest}}\label{\detokenize{test:conftest-module}}\index{module@\spxentry{module}!conftest@\spxentry{conftest}}\index{conftest@\spxentry{conftest}!module@\spxentry{module}}\index{pytest\_addoption() (in module conftest)@\spxentry{pytest\_addoption()}\spxextra{in module conftest}}

\begin{fulllineitems}
\phantomsection\label{\detokenize{test:conftest.pytest_addoption}}\pysiglinewithargsret{\sphinxcode{\sphinxupquote{conftest.}}\sphinxbfcode{\sphinxupquote{pytest\_addoption}}}{\emph{\DUrole{n}{parser}}}{}
\sphinxAtStartPar
Add argument parser to pytest, we can pass parameters to pytest.

\end{fulllineitems}

\index{pytest\_generate\_tests() (in module conftest)@\spxentry{pytest\_generate\_tests()}\spxextra{in module conftest}}

\begin{fulllineitems}
\phantomsection\label{\detokenize{test:conftest.pytest_generate_tests}}\pysiglinewithargsret{\sphinxcode{\sphinxupquote{conftest.}}\sphinxbfcode{\sphinxupquote{pytest\_generate\_tests}}}{\emph{\DUrole{n}{metafunc}}}{}
\sphinxAtStartPar
Convert parser arguments to parameters

\end{fulllineitems}



\subsection{test\_image\_preprocessing module}
\label{\detokenize{test:module-test_image_preprocessing}}\label{\detokenize{test:test-image-preprocessing-module}}\index{module@\spxentry{module}!test\_image\_preprocessing@\spxentry{test\_image\_preprocessing}}\index{test\_image\_preprocessing@\spxentry{test\_image\_preprocessing}!module@\spxentry{module}}\index{DSC() (in module test\_image\_preprocessing)@\spxentry{DSC()}\spxextra{in module test\_image\_preprocessing}}

\begin{fulllineitems}
\phantomsection\label{\detokenize{test:test_image_preprocessing.DSC}}\pysiglinewithargsret{\sphinxcode{\sphinxupquote{test\_image\_preprocessing.}}\sphinxbfcode{\sphinxupquote{DSC}}}{\emph{\DUrole{n}{ref\_image}}, \emph{\DUrole{n}{test\_image}}}{}
\sphinxAtStartPar
Calculate the Dice similarity coefficient
\begin{quote}\begin{description}
\item[{Parameters}] \leavevmode\begin{itemize}
\item {} 
\sphinxAtStartPar
\sphinxstyleliteralstrong{\sphinxupquote{ref\_image}} (\sphinxstyleliteralemphasis{\sphinxupquote{numpy.ndarrays}}) \textendash{} nda to compare

\item {} 
\sphinxAtStartPar
\sphinxstyleliteralstrong{\sphinxupquote{test\_image}} (\sphinxstyleliteralemphasis{\sphinxupquote{numpy.ndarrays}}) \textendash{} nda to compare

\end{itemize}

\item[{Returns}] \leavevmode
\sphinxAtStartPar
The Dice similarity coefficient of the reference and test image

\item[{Return type}] \leavevmode
\sphinxAtStartPar
float

\end{description}\end{quote}

\end{fulllineitems}

\index{get\_cropped\_volume\_image() (in module test\_image\_preprocessing)@\spxentry{get\_cropped\_volume\_image()}\spxextra{in module test\_image\_preprocessing}}

\begin{fulllineitems}
\phantomsection\label{\detokenize{test:test_image_preprocessing.get_cropped_volume_image}}\pysiglinewithargsret{\sphinxcode{\sphinxupquote{test\_image\_preprocessing.}}\sphinxbfcode{\sphinxupquote{get\_cropped\_volume\_image}}}{\emph{\DUrole{n}{testCase}}}{}
\sphinxAtStartPar
Read the cropped volume
\begin{quote}\begin{description}
\item[{Returns}] \leavevmode
\sphinxAtStartPar
The cropped volume sitk image

\item[{Return type}] \leavevmode
\sphinxAtStartPar
SITK

\end{description}\end{quote}

\end{fulllineitems}

\index{mean\_absolute\_error() (in module test\_image\_preprocessing)@\spxentry{mean\_absolute\_error()}\spxextra{in module test\_image\_preprocessing}}

\begin{fulllineitems}
\phantomsection\label{\detokenize{test:test_image_preprocessing.mean_absolute_error}}\pysiglinewithargsret{\sphinxcode{\sphinxupquote{test\_image\_preprocessing.}}\sphinxbfcode{\sphinxupquote{mean\_absolute\_error}}}{\emph{\DUrole{n}{ref\_image}}, \emph{\DUrole{n}{test\_image}}}{}
\end{fulllineitems}

\index{mean\_square\_error() (in module test\_image\_preprocessing)@\spxentry{mean\_square\_error()}\spxextra{in module test\_image\_preprocessing}}

\begin{fulllineitems}
\phantomsection\label{\detokenize{test:test_image_preprocessing.mean_square_error}}\pysiglinewithargsret{\sphinxcode{\sphinxupquote{test\_image\_preprocessing.}}\sphinxbfcode{\sphinxupquote{mean\_square\_error}}}{\emph{\DUrole{n}{ref\_image}}, \emph{\DUrole{n}{test\_image}}}{}
\end{fulllineitems}

\index{print\_result() (in module test\_image\_preprocessing)@\spxentry{print\_result()}\spxextra{in module test\_image\_preprocessing}}

\begin{fulllineitems}
\phantomsection\label{\detokenize{test:test_image_preprocessing.print_result}}\pysiglinewithargsret{\sphinxcode{\sphinxupquote{test\_image\_preprocessing.}}\sphinxbfcode{\sphinxupquote{print\_result}}}{\emph{\DUrole{n}{ref\_image}}, \emph{\DUrole{n}{test\_image}}, \emph{\DUrole{n}{seg\_type}}}{}
\end{fulllineitems}

\index{read\_asc\_volume\_image() (in module test\_image\_preprocessing)@\spxentry{read\_asc\_volume\_image()}\spxextra{in module test\_image\_preprocessing}}

\begin{fulllineitems}
\phantomsection\label{\detokenize{test:test_image_preprocessing.read_asc_volume_image}}\pysiglinewithargsret{\sphinxcode{\sphinxupquote{test\_image\_preprocessing.}}\sphinxbfcode{\sphinxupquote{read\_asc\_volume\_image}}}{\emph{\DUrole{n}{testCase}}}{}
\sphinxAtStartPar
Read the segmented ascending and descending aorta volume
\begin{quote}\begin{description}
\item[{Returns}] \leavevmode
\sphinxAtStartPar
The segmented ascending and descending aorta sitk image

\item[{Return type}] \leavevmode
\sphinxAtStartPar
SITK

\end{description}\end{quote}

\end{fulllineitems}

\index{read\_desc\_volume\_image() (in module test\_image\_preprocessing)@\spxentry{read\_desc\_volume\_image()}\spxextra{in module test\_image\_preprocessing}}

\begin{fulllineitems}
\phantomsection\label{\detokenize{test:test_image_preprocessing.read_desc_volume_image}}\pysiglinewithargsret{\sphinxcode{\sphinxupquote{test\_image\_preprocessing.}}\sphinxbfcode{\sphinxupquote{read\_desc\_volume\_image}}}{\emph{\DUrole{n}{testCase}}}{}
\sphinxAtStartPar
Read the segmented descending aorta volume
\begin{quote}\begin{description}
\item[{Returns}] \leavevmode
\sphinxAtStartPar
The segmented descending aorta sitk image

\item[{Return type}] \leavevmode
\sphinxAtStartPar
SITK

\end{description}\end{quote}

\end{fulllineitems}

\index{read\_final\_volume\_image() (in module test\_image\_preprocessing)@\spxentry{read\_final\_volume\_image()}\spxextra{in module test\_image\_preprocessing}}

\begin{fulllineitems}
\phantomsection\label{\detokenize{test:test_image_preprocessing.read_final_volume_image}}\pysiglinewithargsret{\sphinxcode{\sphinxupquote{test\_image\_preprocessing.}}\sphinxbfcode{\sphinxupquote{read\_final\_volume\_image}}}{\emph{\DUrole{n}{testCase}}}{}
\sphinxAtStartPar
Read the final segmented aorta volume
\begin{quote}\begin{description}
\item[{Returns}] \leavevmode
\sphinxAtStartPar
The final segmented aorta sitk image

\item[{Return type}] \leavevmode
\sphinxAtStartPar
SITK

\end{description}\end{quote}

\end{fulllineitems}

\index{root\_mse() (in module test\_image\_preprocessing)@\spxentry{root\_mse()}\spxextra{in module test\_image\_preprocessing}}

\begin{fulllineitems}
\phantomsection\label{\detokenize{test:test_image_preprocessing.root_mse}}\pysiglinewithargsret{\sphinxcode{\sphinxupquote{test\_image\_preprocessing.}}\sphinxbfcode{\sphinxupquote{root\_mse}}}{\emph{\DUrole{n}{ref\_image}}, \emph{\DUrole{n}{test\_image}}}{}
\end{fulllineitems}

\index{test\_asc\_and\_final() (in module test\_image\_preprocessing)@\spxentry{test\_asc\_and\_final()}\spxextra{in module test\_image\_preprocessing}}

\begin{fulllineitems}
\phantomsection\label{\detokenize{test:test_image_preprocessing.test_asc_and_final}}\pysiglinewithargsret{\sphinxcode{\sphinxupquote{test\_image\_preprocessing.}}\sphinxbfcode{\sphinxupquote{test\_asc\_and\_final}}}{\emph{\DUrole{n}{limit}}, \emph{\DUrole{n}{qualifiedCoef}}, \emph{\DUrole{n}{ffactor}}, \emph{\DUrole{n}{testCase}}, \emph{\DUrole{n}{processing\_image}\DUrole{o}{=}\DUrole{default_value}{None}}}{}
\end{fulllineitems}

\index{test\_compare\_asc() (in module test\_image\_preprocessing)@\spxentry{test\_compare\_asc()}\spxextra{in module test\_image\_preprocessing}}

\begin{fulllineitems}
\phantomsection\label{\detokenize{test:test_image_preprocessing.test_compare_asc}}\pysiglinewithargsret{\sphinxcode{\sphinxupquote{test\_image\_preprocessing.}}\sphinxbfcode{\sphinxupquote{test\_compare\_asc}}}{\emph{\DUrole{n}{limit}}, \emph{\DUrole{n}{qualifiedCoef}}, \emph{\DUrole{n}{ffactor}}, \emph{\DUrole{n}{testCase}}, \emph{\DUrole{n}{processing\_image}\DUrole{o}{=}\DUrole{default_value}{None}}}{}
\sphinxAtStartPar
Read a test cases’ partially segmented volume (descending aorta),
perform ascending aorta segmentation,
and compare the result with the existing ascending aorta volume
\begin{quote}\begin{description}
\item[{Parameters}] \leavevmode\begin{itemize}
\item {} 
\sphinxAtStartPar
\sphinxstyleliteralstrong{\sphinxupquote{limit}} (\sphinxstyleliteralemphasis{\sphinxupquote{float}}) \textendash{} the maximum Dice similarity coefficient difference allowed to pass the test.

\item {} 
\sphinxAtStartPar
\sphinxstyleliteralstrong{\sphinxupquote{qualifiedCoef}} (\sphinxstyleliteralemphasis{\sphinxupquote{float}}) \textendash{} the qualified coefficient value to process descending aorta segmentation.

\item {} 
\sphinxAtStartPar
\sphinxstyleliteralstrong{\sphinxupquote{ffactor}} (\sphinxstyleliteralemphasis{\sphinxupquote{float}}) \textendash{} the factor used to determine the lower and upper threshold for label statistics image filter.

\item {} 
\sphinxAtStartPar
\sphinxstyleliteralstrong{\sphinxupquote{testCase}} (\sphinxstyleliteralemphasis{\sphinxupquote{int}}) \textendash{} the test case to run the test (0\sphinxhyphen{}5).

\end{itemize}

\item[{Returns}] \leavevmode
\sphinxAtStartPar
The Dice similarity coefficient of the reference and test image

\item[{Return type}] \leavevmode
\sphinxAtStartPar
float

\end{description}\end{quote}

\end{fulllineitems}

\index{test\_compare\_des() (in module test\_image\_preprocessing)@\spxentry{test\_compare\_des()}\spxextra{in module test\_image\_preprocessing}}

\begin{fulllineitems}
\phantomsection\label{\detokenize{test:test_image_preprocessing.test_compare_des}}\pysiglinewithargsret{\sphinxcode{\sphinxupquote{test\_image\_preprocessing.}}\sphinxbfcode{\sphinxupquote{test\_compare\_des}}}{\emph{\DUrole{n}{limit}}, \emph{\DUrole{n}{qualifiedCoef}}, \emph{\DUrole{n}{ffactor}}, \emph{\DUrole{n}{testCase}}}{}
\sphinxAtStartPar
Read a test cases’ cropped volume,
perform descending aorta segmentation,
and compare the result with the existing volume
\begin{quote}\begin{description}
\item[{Parameters}] \leavevmode\begin{itemize}
\item {} 
\sphinxAtStartPar
\sphinxstyleliteralstrong{\sphinxupquote{limit}} (\sphinxstyleliteralemphasis{\sphinxupquote{float}}) \textendash{} the maximum Dice similarity coefficient difference allowed to pass the test.

\item {} 
\sphinxAtStartPar
\sphinxstyleliteralstrong{\sphinxupquote{qualifiedCoef}} (\sphinxstyleliteralemphasis{\sphinxupquote{float}}) \textendash{} the qualified coefficient value to process descending aorta segmentation.

\item {} 
\sphinxAtStartPar
\sphinxstyleliteralstrong{\sphinxupquote{ffactor}} (\sphinxstyleliteralemphasis{\sphinxupquote{float}}) \textendash{} the factor used to determine the lower and upper threshold for label statistics image filter.

\item {} 
\sphinxAtStartPar
\sphinxstyleliteralstrong{\sphinxupquote{testCase}} (\sphinxstyleliteralemphasis{\sphinxupquote{int}}) \textendash{} the test case to run the test (0\sphinxhyphen{}5).

\end{itemize}

\item[{Returns}] \leavevmode
\sphinxAtStartPar
The Dice similarity coefficient of the reference and test image

\item[{Return type}] \leavevmode
\sphinxAtStartPar
float

\end{description}\end{quote}

\end{fulllineitems}

\index{test\_compare\_final\_volume() (in module test\_image\_preprocessing)@\spxentry{test\_compare\_final\_volume()}\spxextra{in module test\_image\_preprocessing}}

\begin{fulllineitems}
\phantomsection\label{\detokenize{test:test_image_preprocessing.test_compare_final_volume}}\pysiglinewithargsret{\sphinxcode{\sphinxupquote{test\_image\_preprocessing.}}\sphinxbfcode{\sphinxupquote{test\_compare\_final\_volume}}}{\emph{\DUrole{n}{limit}}, \emph{\DUrole{n}{qualifiedCoef}}, \emph{\DUrole{n}{ffactor}}, \emph{\DUrole{n}{testCase}}, \emph{\DUrole{n}{processing\_image}\DUrole{o}{=}\DUrole{default_value}{None}}}{}
\end{fulllineitems}

\index{test\_prepared\_segmenting\_image() (in module test\_image\_preprocessing)@\spxentry{test\_prepared\_segmenting\_image()}\spxextra{in module test\_image\_preprocessing}}

\begin{fulllineitems}
\phantomsection\label{\detokenize{test:test_image_preprocessing.test_prepared_segmenting_image}}\pysiglinewithargsret{\sphinxcode{\sphinxupquote{test\_image\_preprocessing.}}\sphinxbfcode{\sphinxupquote{test\_prepared\_segmenting\_image}}}{\emph{\DUrole{n}{limit}}, \emph{\DUrole{n}{qualifiedCoef}}, \emph{\DUrole{n}{ffactor}}, \emph{\DUrole{n}{testCase}}}{}
\end{fulllineitems}

\index{transform\_image() (in module test\_image\_preprocessing)@\spxentry{transform\_image()}\spxextra{in module test\_image\_preprocessing}}

\begin{fulllineitems}
\phantomsection\label{\detokenize{test:test_image_preprocessing.transform_image}}\pysiglinewithargsret{\sphinxcode{\sphinxupquote{test\_image\_preprocessing.}}\sphinxbfcode{\sphinxupquote{transform\_image}}}{\emph{\DUrole{n}{cropped\_image}}}{}
\end{fulllineitems}



\subsection{Module contents}
\label{\detokenize{test:module-test}}\label{\detokenize{test:module-contents}}\index{module@\spxentry{module}!test@\spxentry{test}}\index{test@\spxentry{test}!module@\spxentry{module}}

\section{AortaGeomReconDisplayModuleLib package}
\label{\detokenize{AortaGeomReconDisplayModuleLib:aortageomrecondisplaymodulelib-package}}\label{\detokenize{AortaGeomReconDisplayModuleLib::doc}}

\subsection{Submodules}
\label{\detokenize{AortaGeomReconDisplayModuleLib:submodules}}

\subsection{AortaGeomReconDisplayModuleLib.AortaGeomReconEnums module}
\label{\detokenize{AortaGeomReconDisplayModuleLib:module-AortaGeomReconEnums}}\label{\detokenize{AortaGeomReconDisplayModuleLib:aortageomrecondisplaymodulelib-aortageomreconenums-module}}\index{module@\spxentry{module}!AortaGeomReconEnums@\spxentry{AortaGeomReconEnums}}\index{AortaGeomReconEnums@\spxentry{AortaGeomReconEnums}!module@\spxentry{module}}\index{PixelValue (class in AortaGeomReconEnums)@\spxentry{PixelValue}\spxextra{class in AortaGeomReconEnums}}

\begin{fulllineitems}
\phantomsection\label{\detokenize{AortaGeomReconDisplayModuleLib:AortaGeomReconEnums.PixelValue}}\pysiglinewithargsret{\sphinxbfcode{\sphinxupquote{class\DUrole{w}{  }}}\sphinxcode{\sphinxupquote{AortaGeomReconEnums.}}\sphinxbfcode{\sphinxupquote{PixelValue}}}{\emph{\DUrole{n}{value}}}{}
\sphinxAtStartPar
Bases: \sphinxcode{\sphinxupquote{enum.Enum}}

\sphinxAtStartPar
An enumeration.
\index{black\_pixel (AortaGeomReconEnums.PixelValue attribute)@\spxentry{black\_pixel}\spxextra{AortaGeomReconEnums.PixelValue attribute}}

\begin{fulllineitems}
\phantomsection\label{\detokenize{AortaGeomReconDisplayModuleLib:AortaGeomReconEnums.PixelValue.black_pixel}}\pysigline{\sphinxbfcode{\sphinxupquote{black\_pixel}}\sphinxbfcode{\sphinxupquote{\DUrole{w}{  }\DUrole{p}{=}\DUrole{w}{  }0}}}
\end{fulllineitems}

\index{white\_pixel (AortaGeomReconEnums.PixelValue attribute)@\spxentry{white\_pixel}\spxextra{AortaGeomReconEnums.PixelValue attribute}}

\begin{fulllineitems}
\phantomsection\label{\detokenize{AortaGeomReconDisplayModuleLib:AortaGeomReconEnums.PixelValue.white_pixel}}\pysigline{\sphinxbfcode{\sphinxupquote{white\_pixel}}\sphinxbfcode{\sphinxupquote{\DUrole{w}{  }\DUrole{p}{=}\DUrole{w}{  }1}}}
\end{fulllineitems}


\end{fulllineitems}

\index{SegmentDirection (class in AortaGeomReconEnums)@\spxentry{SegmentDirection}\spxextra{class in AortaGeomReconEnums}}

\begin{fulllineitems}
\phantomsection\label{\detokenize{AortaGeomReconDisplayModuleLib:AortaGeomReconEnums.SegmentDirection}}\pysiglinewithargsret{\sphinxbfcode{\sphinxupquote{class\DUrole{w}{  }}}\sphinxcode{\sphinxupquote{AortaGeomReconEnums.}}\sphinxbfcode{\sphinxupquote{SegmentDirection}}}{\emph{\DUrole{n}{value}}}{}
\sphinxAtStartPar
Bases: \sphinxcode{\sphinxupquote{enum.Enum}}

\sphinxAtStartPar
Enum type describing the segmentation direction.
Superior is where the human head is located.
Inferior is where the human feet is located.
\index{Inferior\_to\_Superior (AortaGeomReconEnums.SegmentDirection attribute)@\spxentry{Inferior\_to\_Superior}\spxextra{AortaGeomReconEnums.SegmentDirection attribute}}

\begin{fulllineitems}
\phantomsection\label{\detokenize{AortaGeomReconDisplayModuleLib:AortaGeomReconEnums.SegmentDirection.Inferior_to_Superior}}\pysigline{\sphinxbfcode{\sphinxupquote{Inferior\_to\_Superior}}\sphinxbfcode{\sphinxupquote{\DUrole{w}{  }\DUrole{p}{=}\DUrole{w}{  }2}}}
\end{fulllineitems}

\index{Superior\_to\_Inferior (AortaGeomReconEnums.SegmentDirection attribute)@\spxentry{Superior\_to\_Inferior}\spxextra{AortaGeomReconEnums.SegmentDirection attribute}}

\begin{fulllineitems}
\phantomsection\label{\detokenize{AortaGeomReconDisplayModuleLib:AortaGeomReconEnums.SegmentDirection.Superior_to_Inferior}}\pysigline{\sphinxbfcode{\sphinxupquote{Superior\_to\_Inferior}}\sphinxbfcode{\sphinxupquote{\DUrole{w}{  }\DUrole{p}{=}\DUrole{w}{  }1}}}
\end{fulllineitems}


\end{fulllineitems}

\index{SegmentType (class in AortaGeomReconEnums)@\spxentry{SegmentType}\spxextra{class in AortaGeomReconEnums}}

\begin{fulllineitems}
\phantomsection\label{\detokenize{AortaGeomReconDisplayModuleLib:AortaGeomReconEnums.SegmentType}}\pysiglinewithargsret{\sphinxbfcode{\sphinxupquote{class\DUrole{w}{  }}}\sphinxcode{\sphinxupquote{AortaGeomReconEnums.}}\sphinxbfcode{\sphinxupquote{SegmentType}}}{\emph{\DUrole{n}{value}}}{}
\sphinxAtStartPar
Bases: \sphinxcode{\sphinxupquote{enum.Enum}}

\sphinxAtStartPar
An enumeration.
\index{ascending\_aorta (AortaGeomReconEnums.SegmentType attribute)@\spxentry{ascending\_aorta}\spxextra{AortaGeomReconEnums.SegmentType attribute}}

\begin{fulllineitems}
\phantomsection\label{\detokenize{AortaGeomReconDisplayModuleLib:AortaGeomReconEnums.SegmentType.ascending_aorta}}\pysigline{\sphinxbfcode{\sphinxupquote{ascending\_aorta}}\sphinxbfcode{\sphinxupquote{\DUrole{w}{  }\DUrole{p}{=}\DUrole{w}{  }2}}}
\end{fulllineitems}

\index{descending\_aorta (AortaGeomReconEnums.SegmentType attribute)@\spxentry{descending\_aorta}\spxextra{AortaGeomReconEnums.SegmentType attribute}}

\begin{fulllineitems}
\phantomsection\label{\detokenize{AortaGeomReconDisplayModuleLib:AortaGeomReconEnums.SegmentType.descending_aorta}}\pysigline{\sphinxbfcode{\sphinxupquote{descending\_aorta}}\sphinxbfcode{\sphinxupquote{\DUrole{w}{  }\DUrole{p}{=}\DUrole{w}{  }1}}}
\end{fulllineitems}

\index{is\_axial\_seg() (AortaGeomReconEnums.SegmentType method)@\spxentry{is\_axial\_seg()}\spxextra{AortaGeomReconEnums.SegmentType method}}

\begin{fulllineitems}
\phantomsection\label{\detokenize{AortaGeomReconDisplayModuleLib:AortaGeomReconEnums.SegmentType.is_axial_seg}}\pysiglinewithargsret{\sphinxbfcode{\sphinxupquote{is\_axial\_seg}}}{}{}
\sphinxAtStartPar
Return True if the segmentation type is
descending or ascending aorta segmentation, False otherwise.

\end{fulllineitems}

\index{is\_sagittal\_seg() (AortaGeomReconEnums.SegmentType method)@\spxentry{is\_sagittal\_seg()}\spxextra{AortaGeomReconEnums.SegmentType method}}

\begin{fulllineitems}
\phantomsection\label{\detokenize{AortaGeomReconDisplayModuleLib:AortaGeomReconEnums.SegmentType.is_sagittal_seg}}\pysiglinewithargsret{\sphinxbfcode{\sphinxupquote{is\_sagittal\_seg}}}{}{}
\end{fulllineitems}

\index{sagittal (AortaGeomReconEnums.SegmentType attribute)@\spxentry{sagittal}\spxextra{AortaGeomReconEnums.SegmentType attribute}}

\begin{fulllineitems}
\phantomsection\label{\detokenize{AortaGeomReconDisplayModuleLib:AortaGeomReconEnums.SegmentType.sagittal}}\pysigline{\sphinxbfcode{\sphinxupquote{sagittal}}\sphinxbfcode{\sphinxupquote{\DUrole{w}{  }\DUrole{p}{=}\DUrole{w}{  }4}}}
\end{fulllineitems}

\index{sagittal\_front (AortaGeomReconEnums.SegmentType attribute)@\spxentry{sagittal\_front}\spxextra{AortaGeomReconEnums.SegmentType attribute}}

\begin{fulllineitems}
\phantomsection\label{\detokenize{AortaGeomReconDisplayModuleLib:AortaGeomReconEnums.SegmentType.sagittal_front}}\pysigline{\sphinxbfcode{\sphinxupquote{sagittal\_front}}\sphinxbfcode{\sphinxupquote{\DUrole{w}{  }\DUrole{p}{=}\DUrole{w}{  }3}}}
\end{fulllineitems}


\end{fulllineitems}



\subsection{AortaGeomReconDisplayModuleLib.AortaSagitalSegmenter module}
\label{\detokenize{AortaGeomReconDisplayModuleLib:module-AortaSagitalSegmenter}}\label{\detokenize{AortaGeomReconDisplayModuleLib:aortageomrecondisplaymodulelib-aortasagitalsegmenter-module}}\index{module@\spxentry{module}!AortaSagitalSegmenter@\spxentry{AortaSagitalSegmenter}}\index{AortaSagitalSegmenter@\spxentry{AortaSagitalSegmenter}!module@\spxentry{module}}\index{AortaSagitalSegmenter (class in AortaSagitalSegmenter)@\spxentry{AortaSagitalSegmenter}\spxextra{class in AortaSagitalSegmenter}}

\begin{fulllineitems}
\phantomsection\label{\detokenize{AortaGeomReconDisplayModuleLib:AortaSagitalSegmenter.AortaSagitalSegmenter}}\pysiglinewithargsret{\sphinxbfcode{\sphinxupquote{class\DUrole{w}{  }}}\sphinxcode{\sphinxupquote{AortaSagitalSegmenter.}}\sphinxbfcode{\sphinxupquote{AortaSagitalSegmenter}}}{\emph{\DUrole{n}{qualified\_coef}}, \emph{\DUrole{n}{processing\_image}}, \emph{\DUrole{n}{cropped\_image}}}{}
\sphinxAtStartPar
Bases: \sphinxcode{\sphinxupquote{object}}

\sphinxAtStartPar
This class performs Aorta Sagital segmentation
\index{\_\_segment\_sag() (AortaSagitalSegmenter.AortaSagitalSegmenter method)@\spxentry{\_\_segment\_sag()}\spxextra{AortaSagitalSegmenter.AortaSagitalSegmenter method}}

\begin{fulllineitems}
\phantomsection\label{\detokenize{AortaGeomReconDisplayModuleLib:AortaSagitalSegmenter.AortaSagitalSegmenter.__segment_sag}}\pysiglinewithargsret{\sphinxbfcode{\sphinxupquote{\_\_segment\_sag}}}{\emph{\DUrole{n}{sliceNum}}, \emph{\DUrole{n}{factor}}, \emph{\DUrole{n}{size\_factor}}, \emph{\DUrole{n}{current\_size}}, \emph{\DUrole{n}{imgSlice}}, \emph{\DUrole{n}{axial\_seg}}, \emph{\DUrole{n}{seg\_type}}}{}
\end{fulllineitems}

\index{begin\_segmentation() (AortaSagitalSegmenter.AortaSagitalSegmenter method)@\spxentry{begin\_segmentation()}\spxextra{AortaSagitalSegmenter.AortaSagitalSegmenter method}}

\begin{fulllineitems}
\phantomsection\label{\detokenize{AortaGeomReconDisplayModuleLib:AortaSagitalSegmenter.AortaSagitalSegmenter.begin_segmentation}}\pysiglinewithargsret{\sphinxbfcode{\sphinxupquote{begin\_segmentation}}}{}{}
\end{fulllineitems}


\end{fulllineitems}



\subsection{AortaGeomReconDisplayModuleLib.AortaSegmenter module}
\label{\detokenize{AortaGeomReconDisplayModuleLib:module-AortaSegmenter}}\label{\detokenize{AortaGeomReconDisplayModuleLib:aortageomrecondisplaymodulelib-aortasegmenter-module}}\index{module@\spxentry{module}!AortaSegmenter@\spxentry{AortaSegmenter}}\index{AortaSegmenter@\spxentry{AortaSegmenter}!module@\spxentry{module}}\index{AortaSegmenter (class in AortaSegmenter)@\spxentry{AortaSegmenter}\spxextra{class in AortaSegmenter}}

\begin{fulllineitems}
\phantomsection\label{\detokenize{AortaGeomReconDisplayModuleLib:AortaSegmenter.AortaSegmenter}}\pysiglinewithargsret{\sphinxbfcode{\sphinxupquote{class\DUrole{w}{  }}}\sphinxcode{\sphinxupquote{AortaSegmenter.}}\sphinxbfcode{\sphinxupquote{AortaSegmenter}}}{\emph{\DUrole{n}{cropped\_image}}, \emph{\DUrole{n}{starting\_slice}}, \emph{\DUrole{n}{aorta\_centre}}, \emph{\DUrole{n}{num\_slice\_skipping}}, \emph{\DUrole{n}{processing\_image}}, \emph{\DUrole{n}{seg\_type}}, \emph{\DUrole{n}{qualified\_coef}\DUrole{o}{=}\DUrole{default_value}{2.2}}, \emph{\DUrole{n}{filter\_factor}\DUrole{o}{=}\DUrole{default_value}{3.5}}}{}
\sphinxAtStartPar
Bases: \sphinxcode{\sphinxupquote{object}}
\index{\_\_count\_pixel\_asc() (AortaSegmenter.AortaSegmenter method)@\spxentry{\_\_count\_pixel\_asc()}\spxextra{AortaSegmenter.AortaSegmenter method}}

\begin{fulllineitems}
\phantomsection\label{\detokenize{AortaGeomReconDisplayModuleLib:AortaSegmenter.AortaSegmenter.__count_pixel_asc}}\pysiglinewithargsret{\sphinxbfcode{\sphinxupquote{\_\_count\_pixel\_asc}}}{\emph{\DUrole{n}{new\_slice}}}{}
\sphinxAtStartPar
Use label statistics to calculate the number of counted pixels for ascending aorta segmentation.
\begin{quote}\begin{description}
\item[{Returns}] \leavevmode
\sphinxAtStartPar
\begin{description}
\item[{tuple containing:}] \leavevmode
\sphinxAtStartPar
int: The total number of the X coordinates where it is white pixel
tupple: The new derived centre calculated by the mean of X coordinates and Y coordinates where it is white pixel
list: The new seeds coordinates based on the new derived centre

\end{description}


\item[{Return type}] \leavevmode
\sphinxAtStartPar
(tuple)

\end{description}\end{quote}

\end{fulllineitems}

\index{\_\_count\_pixel\_des() (AortaSegmenter.AortaSegmenter method)@\spxentry{\_\_count\_pixel\_des()}\spxextra{AortaSegmenter.AortaSegmenter method}}

\begin{fulllineitems}
\phantomsection\label{\detokenize{AortaGeomReconDisplayModuleLib:AortaSegmenter.AortaSegmenter.__count_pixel_des}}\pysiglinewithargsret{\sphinxbfcode{\sphinxupquote{\_\_count\_pixel\_des}}}{\emph{\DUrole{n}{new\_slice}}}{}
\sphinxAtStartPar
Use label statistics to calculate the number of counted pixels for descending aorta segmentation.
\begin{quote}\begin{description}
\item[{Returns}] \leavevmode
\sphinxAtStartPar
\begin{description}
\item[{tuple containing:}] \leavevmode
\sphinxAtStartPar
int: The total number of the counted X coordinates
tupple: The new derived centre calculated by the mean of counted X coordinates and Y coordinates

\end{description}


\item[{Return type}] \leavevmode
\sphinxAtStartPar
(tuple)

\end{description}\end{quote}

\end{fulllineitems}

\index{\_\_filling\_missing\_slices() (AortaSegmenter.AortaSegmenter method)@\spxentry{\_\_filling\_missing\_slices()}\spxextra{AortaSegmenter.AortaSegmenter method}}

\begin{fulllineitems}
\phantomsection\label{\detokenize{AortaGeomReconDisplayModuleLib:AortaSegmenter.AortaSegmenter.__filling_missing_slices}}\pysiglinewithargsret{\sphinxbfcode{\sphinxupquote{\_\_filling\_missing\_slices}}}{}{}
\sphinxAtStartPar
The helper function to replace the missing slice that was not accepted during the descending aorta segmentation.
This function will replace the missing slice by reading the previous slice and the next slice,
fill the slice with the overlapping area of both slices.

\end{fulllineitems}

\index{\_\_get\_label\_statistics() (AortaSegmenter.AortaSegmenter method)@\spxentry{\_\_get\_label\_statistics()}\spxextra{AortaSegmenter.AortaSegmenter method}}

\begin{fulllineitems}
\phantomsection\label{\detokenize{AortaGeomReconDisplayModuleLib:AortaSegmenter.AortaSegmenter.__get_label_statistics}}\pysiglinewithargsret{\sphinxbfcode{\sphinxupquote{\_\_get\_label\_statistics}}}{}{}
\sphinxAtStartPar
Use SITK::SignedMaurerDistanceMap to calculate the Euclidean distance transform
and use SITK::LabelStatisticsImageFilter to derive descriptive intensity.
\begin{quote}\begin{description}
\item[{Returns}] \leavevmode
\sphinxAtStartPar
labeled statistics of the original image.

\item[{Return type}] \leavevmode
\sphinxAtStartPar
numpy.ndarray

\end{description}\end{quote}

\end{fulllineitems}

\index{\_\_is\_new\_centre\_qualified() (AortaSegmenter.AortaSegmenter method)@\spxentry{\_\_is\_new\_centre\_qualified()}\spxextra{AortaSegmenter.AortaSegmenter method}}

\begin{fulllineitems}
\phantomsection\label{\detokenize{AortaGeomReconDisplayModuleLib:AortaSegmenter.AortaSegmenter.__is_new_centre_qualified}}\pysiglinewithargsret{\sphinxbfcode{\sphinxupquote{\_\_is\_new\_centre\_qualified}}}{\emph{\DUrole{n}{total\_coord}}, \emph{\DUrole{n}{is\_overlapping}}}{}
\sphinxAtStartPar
Return True if the number of coordiante in the segmented centre is qualified
\begin{quote}\begin{description}
\item[{Returns}] \leavevmode
\sphinxAtStartPar
Boolean

\end{description}\end{quote}

\end{fulllineitems}

\index{\_\_is\_overlapping() (AortaSegmenter.AortaSegmenter method)@\spxentry{\_\_is\_overlapping()}\spxextra{AortaSegmenter.AortaSegmenter method}}

\begin{fulllineitems}
\phantomsection\label{\detokenize{AortaGeomReconDisplayModuleLib:AortaSegmenter.AortaSegmenter.__is_overlapping}}\pysiglinewithargsret{\sphinxbfcode{\sphinxupquote{\_\_is\_overlapping}}}{\emph{\DUrole{n}{img1}}, \emph{\DUrole{n}{i}}}{}
\sphinxAtStartPar
Compare the current segmented slice with the next two slices,
return True if any overlaps otherwise False
\begin{quote}\begin{description}
\item[{Returns}] \leavevmode
\sphinxAtStartPar
comparison result

\item[{Return type}] \leavevmode
\sphinxAtStartPar
Boolean

\end{description}\end{quote}

\end{fulllineitems}

\index{\_\_prepare\_seed() (AortaSegmenter.AortaSegmenter method)@\spxentry{\_\_prepare\_seed()}\spxextra{AortaSegmenter.AortaSegmenter method}}

\begin{fulllineitems}
\phantomsection\label{\detokenize{AortaGeomReconDisplayModuleLib:AortaSegmenter.AortaSegmenter.__prepare_seed}}\pysiglinewithargsret{\sphinxbfcode{\sphinxupquote{\_\_prepare\_seed}}}{}{}
\sphinxAtStartPar
Get a seed from the original image. We will add extra space
and use it to get the labeled image statistics.
\begin{quote}\begin{description}
\item[{Returns}] \leavevmode
\sphinxAtStartPar
An image slice with aorta centre and some extra spacing.

\item[{Return type}] \leavevmode
\sphinxAtStartPar
SITK::IMAGE

\end{description}\end{quote}

\end{fulllineitems}

\index{\_\_segmentation() (AortaSegmenter.AortaSegmenter method)@\spxentry{\_\_segmentation()}\spxextra{AortaSegmenter.AortaSegmenter method}}

\begin{fulllineitems}
\phantomsection\label{\detokenize{AortaGeomReconDisplayModuleLib:AortaSegmenter.AortaSegmenter.__segmentation}}\pysiglinewithargsret{\sphinxbfcode{\sphinxupquote{\_\_segmentation}}}{}{}
\sphinxAtStartPar
From the starting slice to the superior or the inferior,
use label statistics to see if a circle can be segmented.

\end{fulllineitems}

\index{begin\_segmentation() (AortaSegmenter.AortaSegmenter method)@\spxentry{begin\_segmentation()}\spxextra{AortaSegmenter.AortaSegmenter method}}

\begin{fulllineitems}
\phantomsection\label{\detokenize{AortaGeomReconDisplayModuleLib:AortaSegmenter.AortaSegmenter.begin_segmentation}}\pysiglinewithargsret{\sphinxbfcode{\sphinxupquote{begin\_segmentation}}}{}{}
\sphinxAtStartPar
The public method to process segmentation.

\end{fulllineitems}

\index{cropped\_image (AortaSegmenter.AortaSegmenter property)@\spxentry{cropped\_image}\spxextra{AortaSegmenter.AortaSegmenter property}}

\begin{fulllineitems}
\phantomsection\label{\detokenize{AortaGeomReconDisplayModuleLib:AortaSegmenter.AortaSegmenter.cropped_image}}\pysigline{\sphinxbfcode{\sphinxupquote{property\DUrole{w}{  }}}\sphinxbfcode{\sphinxupquote{cropped\_image}}}
\end{fulllineitems}

\index{processing\_image (AortaSegmenter.AortaSegmenter property)@\spxentry{processing\_image}\spxextra{AortaSegmenter.AortaSegmenter property}}

\begin{fulllineitems}
\phantomsection\label{\detokenize{AortaGeomReconDisplayModuleLib:AortaSegmenter.AortaSegmenter.processing_image}}\pysigline{\sphinxbfcode{\sphinxupquote{property\DUrole{w}{  }}}\sphinxbfcode{\sphinxupquote{processing\_image}}}
\end{fulllineitems}


\end{fulllineitems}



\subsection{Module contents}
\label{\detokenize{AortaGeomReconDisplayModuleLib:module-AortaGeomReconDisplayModuleLib}}\label{\detokenize{AortaGeomReconDisplayModuleLib:module-contents}}\index{module@\spxentry{module}!AortaGeomReconDisplayModuleLib@\spxentry{AortaGeomReconDisplayModuleLib}}\index{AortaGeomReconDisplayModuleLib@\spxentry{AortaGeomReconDisplayModuleLib}!module@\spxentry{module}}

\chapter{Indices and tables}
\label{\detokenize{index:indices-and-tables}}
\begin{DUlineblock}{0em}
\item[] \DUrole{xref,std,std-ref}{genindex}
\item[] \DUrole{xref,std,std-ref}{modindex}
\end{DUlineblock}


\section{Glossary of Terms Used in AortaGeomRecon Documentation}
\label{\detokenize{glossary:glossary-of-terms-used-in-aortageomrecon-documentation}}\label{\detokenize{glossary::doc}}\begin{description}
\item[{Aorta\index{Aorta@\spxentry{Aorta}|spxpagem}\phantomsection\label{\detokenize{glossary:term-Aorta}}}] \leavevmode
\sphinxAtStartPar
The aorta is the largest artery of the body and carries blood from the heart to the circulatory system. It has several sections: The Aortic Root, the transition point where blood first exits the heart, functions as the water main of the body.

\item[{Organ Segmentation\index{Organ Segmentation@\spxentry{Organ Segmentation}|spxpagem}\phantomsection\label{\detokenize{glossary:term-Organ-Segmentation}}}] \leavevmode
\sphinxAtStartPar
The definition of the organ boundary or the organ segmentation is helpful for orientation and identification of the regions of interests inside the organ during the diagnostic or treatment procedure. Further, it allows the volume estimation of the organ.

\item[{3D Slicer\index{3D Slicer@\spxentry{3D Slicer}|spxpagem}\phantomsection\label{\detokenize{glossary:term-3D-Slicer}}}] \leavevmode
\sphinxAtStartPar
3D \sphinxhref{https://www.slicer.org/}{Slicer} (Slicer) is a free and open source software package for image analysis and scientific visualization. Slicer is used in a variety of medical applications, including autism, multiple sclerosis, systemic lupus erythematosus, prostate cancer, lung cancer, breast cancer, schizophrenia, orthopedic biomechanics, COPD, cardiovascular disease and neurosurgery.

\item[{Descending Aorta\index{Descending Aorta@\spxentry{Descending Aorta}|spxpagem}\phantomsection\label{\detokenize{glossary:term-Descending-Aorta}}}] \leavevmode
\sphinxAtStartPar
The descending aorta is the longest part of your aorta (the largest artery in your body). It begins after your left subclavian artery branches from your aortic arch, and it extends downward into your belly.

\item[{Ascending Aorta\index{Ascending Aorta@\spxentry{Ascending Aorta}|spxpagem}\phantomsection\label{\detokenize{glossary:term-Ascending-Aorta}}}] \leavevmode
\sphinxAtStartPar
The ascending aorta is the first part of the aorta, which is the largest blood vessel in your body. It comes out of your heart and pumps blood through the aortic arch and into the descending aorta.

\item[{Qualified coefficient\index{Qualified coefficient@\spxentry{Qualified coefficient}|spxpagem}\phantomsection\label{\detokenize{glossary:term-Qualified-coefficient}}}] \leavevmode
\sphinxAtStartPar
The coefficient used in deteriming the threshold for labelstatisticsimagefilter

\end{description}


\renewcommand{\indexname}{Python Module Index}
\begin{sphinxtheindex}
\let\bigletter\sphinxstyleindexlettergroup
\bigletter{a}
\item\relax\sphinxstyleindexentry{AortaGeomReconDisplayModuleLib}\sphinxstyleindexpageref{AortaGeomReconDisplayModuleLib:\detokenize{module-AortaGeomReconDisplayModuleLib}}
\item\relax\sphinxstyleindexentry{AortaGeomReconEnums}\sphinxstyleindexpageref{AortaGeomReconDisplayModuleLib:\detokenize{module-AortaGeomReconEnums}}
\item\relax\sphinxstyleindexentry{AortaSagitalSegmenter}\sphinxstyleindexpageref{AortaGeomReconDisplayModuleLib:\detokenize{module-AortaSagitalSegmenter}}
\item\relax\sphinxstyleindexentry{AortaSegmenter}\sphinxstyleindexpageref{AortaGeomReconDisplayModuleLib:\detokenize{module-AortaSegmenter}}
\indexspace
\bigletter{c}
\item\relax\sphinxstyleindexentry{conftest}\sphinxstyleindexpageref{test:\detokenize{module-conftest}}
\indexspace
\bigletter{t}
\item\relax\sphinxstyleindexentry{test}\sphinxstyleindexpageref{test:\detokenize{module-test}}
\item\relax\sphinxstyleindexentry{test\_image\_preprocessing}\sphinxstyleindexpageref{test:\detokenize{module-test_image_preprocessing}}
\end{sphinxtheindex}

\renewcommand{\indexname}{Index}
\printindex
\end{document}